\subsection{Introduction}
Sonar is one of the primary sensing modalities available in underwater and near surface environments, where visual sensors suffer from turbidity, light attenuation, and poor visibility. To make effective use of sonar measurements in a SLAM pipeline, it is necessary to understand the basic physics behind acoustic wave propagation and how sonar sensors generate, transmit, and receive these signals. This section provides a brief and practical overview of the underlying principles, serving only as a foundation for the later chapters where sonar data is processed, normalized, and converted into local and global maps. The goal is not to dive deeply into acoustics or advanced modeling, but to establish a clear understanding of how sonar interacts with the environment and why its characteristics matter for mapping.
\begin{figure}[H]
    \centering
    \includegraphics[width=1.0\linewidth]{Pictures/Sonar_Theory/Introduction/Active_vs_Passive_Sonar.jpg}
    \caption{Picture of difference between active and passive sonar, both types can be used for marine navigation, however active is the most used and flexible method for marine navigation with sonar.\textsuperscript{\cite{sonar_active_vs_passive}}}
    \label{fig:sonar-theory-sonar-active-vs-passive}
\end{figure}

