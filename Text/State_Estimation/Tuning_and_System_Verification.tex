\subsection{Tuning and System Verification}
\begin{itemize}
    \item Describe the role of process ($Q$) and measurement ($R$) noise covariance tuning.
    \item Explain methods such as innovation monitoring and adaptive noise estimation.
    \item Introduce the Normalized Innovation Squared (NIS) test for filter consistency validation.
    \item Mention residual monitoring and gating for outlier rejection.
    \item Summarize empirical tuning steps performed using real-world test data.
\end{itemize}

Final State Definition maybe just shortly mantion at the very end here nothing much jesjes and mention that UKF-M is the best way forward
\begin{itemize}
    \item Present the final ESKF state vector:
    $$
        \mathbf{x} =
        \begin{bmatrix}
            \mathbf{p} & \mathbf{v} & \mathbf{q} & \mathbf{b}_a & \mathbf{b}_g
        \end{bmatrix}^\top
    $$
    \item Explain that position, velocity, and attitude are expressed in the navigation frame, while accelerometer and gyroscope biases evolve in the body frame.
    \item Conclude that this representation provides an optimal balance between model realism, computational efficiency, and estimator robustness.
\end{itemize}