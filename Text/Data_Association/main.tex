\section{Data Association}


Gating (Mahalanobis)

matching/tracking

Also talk about loop closures

produce measurement factors for optimizer

DA - Mixture Models <3<3<3<3<3 JCBB <3<3<3 or RANSAC <3 or MTH and probabalistic techniques (eghhhh questionable)
Ideal using MHT (Multi Hypothesis Theory) => Very robust => However it is very slow, so slow that there is no use case exept of offline uses like simulation and post processing and post analysis. so not a viable strart.
probabalistic techniques:
  - posterior probability hypothesis (very expensive) 
  - Variations of JPDA (questionable validity)
  - Particle filter (FastSLAM) (it kind of works, implentation issues and has its own isdues that in practice have shown to be not a suitable solution for SLAM)

This JCBB and RANSAC perfect for stationary objects, but if moving objects it hard, this type of hybrid DA methods still under reaserch... luckily underwater arent manny moving large objects, most are stationary and thus this problem can be avoided for the most part.... some caveats jesjes

Landmark measuremnet factor generation

talk about why stayic or relatively static consideration. because tracking targets and mapping environment is super hard and this field is still under very much reaserch for new methods, so DA algorithms concidered here are for near static environments. Witch is fine as bom of the sea floor changes relatively little. also minor changes in environment can be corrected for using robust estimators witch will be briefly duscuseed in optimizer chaper under iSAM2 beyond Gausian assumptions chaper. however higly dynamic environments these algorithms would struggle alone, however under the sea that isn't to big of an issue. so it will sufice with RANSAC and or JCBB


UPDATE: Mixture Models is the best solution, just make factors between the landmarks as a mixture and then put into factor graph where it optimizes stuff automatically O-O + also suported by GTSAM jippyyyyy :)))

UPDATE: It seems its not realtime MMSAM is not realtime enough X-X


So chapter sugestions:
- Introduction
- Mahalanobis Distance
- NN
- PDA
- RANSAC
- JCBB
- Mixture Models 
- Discussion







======================================








into chaper
start with GATING chaper (
  mahalanobis in there, define mahalanobis distance as 
  $||x-mu||_P=sqrt((x-mu)^T*P^(-1)*(x-mu))$
  this will later be cited in iSAM part
)
then NN
Then probabalistic methods 
(
  discuss PDA, JIPDA, MHT, just main stuff no need to go into math as they will not be used
)
Then RANSAC
then JCBB
Problem of Loop Closure
(
  explain that one could run fastJCBB globally, but the shit would explode exponentially, so instead to make it real time need local (filter DA) vs global (loop closure DA) approach. this means the thing is fast.
  but now question is how do we know when there is a loop closure as we no longer have global DA, need a different way to keep track of global map in a more effective way than DA. Something that doesn't need to be precise, just fast, so we might get false global assumptions but then the global DA can handle those, at least we will not be running global DA every time if we are smart about it and design a method that doesn't have to many false positives to keep it real time.
  explain method for real time detecting loop closures in feature based SLAM-> use comparison methods for old landmarks to new ones to know when we are close somewhere.
  some similarity check 
  keep all 2D frames of sonar old
  perform feature matching algo or similarity checks (We use Descriptor-based matching (more robust) => nearest-neighbor search (NOT nearest neoghbout as in Data Asociation, just nearest value we can find, if it find a value big eough to a set thershold, trigger lol, important to menation this!!!) => (KD-tree) on descriptor vectors for fast compare, use something like OpenCV library) FINAL USE !!!!!!!FLANN Matcher!!!!!!!!!! ALGORITHM! For thsi mention we can use MRPT "nanoflann" function lib to make our lives easier.
  Also perform periodic triggers just in case to be safe that just check periodically if loop closed lol
  score each comparison
  if 1 comparison a lot bigger score then the rest OR a set threshold has been reached -> then we have loop closure 
  ALSO have a failsafe trigger, worst case have a periodic trigger that triggers each 50 steps or something to check if loop closure just to be sure if the other thing doesn't trigger lol.

  BUT also we cant check for the newest data because most likely we will math with newest data and we dont want global gating and global DA for that, that should have local gating, so the newest some amount of new data shoudl be ignored should startet maybe at 2D sonar images that are 1-5 Local Map objects back, idk maybe???? but then how do we handle if we know that its local or not. This is a bad idea gain so the previous targets detected tarcking them would be a pain in the ass os some other smarter way idk man its just weird :/
  THIUS MUST BE ADRESSED!!!
  SOLUTION:
  If you want a concrete implementation rule:
    -Keep a queue of the last N local submaps (e.g., last 5 sonar frames).
    -Local DA operates only on these.
    -Loop closure DA (global) is triggered if similarity > threshold with any submap older than N.
  So when a new measurement comes in:
    1)Check similarity against all older submaps beyond the local window (using FLANN + descriptor scoring).
    2)If score > threshold → run global gating + JCBB on that region.
    3)Otherwise → perform local gating + JCBB on recent submaps only.

  ALso landmark Revalidation: When an old landmark is successfully reobserved and confirmed through JCBB, it is reclassified as a new landmark, effectively resetting its age in the map.
)
THEN Integrated DA framework for SLAM
(
  Each scan:
    check trigger = (similarity checks (FLANN Matcher) + AND Perdioic trigger just in case + only look at features older than N lifespan to ensure we are not looking at a new feature)
    if loop trigger condition true:
      run large gating (global) + JCBB on old map regioon to verify if it's actually loop closure or not lol
    else:
      run small gating (local) + JCBB

    landmark Revalidation: When an old landmark is successfully reobserved and confirmed through JCBB, it is reclassified as a new landmark, effectively resetting its age in the map, so now its in the N lifespan as a young again, so next time we check mathing, it will not check for thsi landmark until it has aged.

    Age young landmarks that are still young. The landmarks that are already older than N dont need aging, just ignore them
)
then factor construction 
(
  Factor graph equation r = h(x, m) - z jesjes something like that with an R somehow jesjes O-O

  Each time data association identifies a new or previously seen landmark, it triggers a new event in the factor graph. The IMU preintegration is paused, a new odometry pose node is created, and a preintegrated IMU factor is added to link it with the previous pose. If the landmark is new, a new landmark node and corresponding observation factor are added; if it is a reobserved landmark, a new factor connects the current pose to the existing landmark node. Loop closures are handled similarly, creating a new pose node and adding factors that reconnect to previously mapped landmarks, ensuring global consistency in the graph.
)