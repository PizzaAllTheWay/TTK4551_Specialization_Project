\subsection{Introduction}
Data association (DA) decides whether a new landmark observation corresponds to an existing map feature or should create a new one, making it the first and most critical gatekeeper in SLAM since every accepted match becomes a trusted factor in the graph \cite{DA_review}. Because even a single wrong match can collapse the entire estimate, the system relies on strict statistical filtering through Mahalanobis gating before any higher level decision is made. Gating defines the admissible uncertainty region in measurement space and controls which candidates enter the DA pipeline at all.
\\ \\
Most SLAM systems assume static or near static environments because reliably handling moving objects remains an open research challenge. Several approaches exist, robust estimators, non Gaussian noise models, motion removal, or explicit tracking, but none are universally reliable. For this thesis the seabed is treated as near static over short and medium time spans, which suits underwater mapping where sediment and structures change slowly. This simplifies DA to a geometric and probabilistic matching problem built on the extended range bearing model used throughout the thesis.
\\ \\
Every association that passes gating becomes a factor linking the current pose to a landmark, meaning the backend optimizer fully trusts the decision. To avoid catastrophic drift, the system relies solely on JCBB, using it in two modes depending on the situation. In ordinary local operation, tight gating leaves only 1 or very few candidates, causing JCBB to collapse naturally into a simple nearest neighbour outcome without performing full joint checks. In ambiguous cases or during loop closure attempts, JCBB switches to its full mode and evaluates joint compatibility using the full SLAM covariance, ensuring that only geometrically consistent multi landmark associations are accepted. Descriptor based loop closure triggering decides when this expensive global JCBB evaluation is necessary and when the lightweight local mode is sufficient.
\\ \\
Although many alternative DA frameworks exist, probabilistic methods like PDA/JIPDA, multi-hypothesis tracking, and mixture model formulations, they are too slow or fragile for real-time SSS SLAM and are therefore only explored but not used in this thesis. The adopted pipeline instead couples gating, JCBB, and descriptor based loop closure detection into a unified scheme that in theory remains computationally efficient while still robust against false matches. This chapter outlines this DA strategy and motivates why these choices are best suited for reliable underwater SLAM.

