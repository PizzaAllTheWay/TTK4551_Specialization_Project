\subsection{Integrated Data Association Framework for SLAM}
\begin{itemize}
    \item For each new scan:
    \begin{itemize}
        \item Perform similarity-based loop closure trigger using FLANN on old submaps (older than $N$).
        \item Also check periodic trigger.
    \end{itemize}
    \item If loop closure suspected:
    \begin{itemize}
        \item Run global gating + JCBB on old region to confirm.
    \end{itemize}
    \item Otherwise:
    \begin{itemize}
        \item Run local gating + JCBB on recent $N$ submaps.
    \end{itemize}
    \item Landmark aging:
    \begin{itemize}
        \item Young landmarks age each scan.
        \item When a landmark passes age $N$, it moves into loop-closure candidate set.
    \end{itemize}
    \item Landmark revalidation:
    \begin{itemize}
        \item When reobserved and passing JCBB, reset to young again.
    \end{itemize}
\end{itemize}
Each scan:
check trigger = (similarity checks (FLANN Matcher) + AND Perdioic trigger just in case + only look at features older than N lifespan to ensure we are not looking at a new feature)
if loop trigger condition true:
run large gating (global) + JCBB on old map regioon to verify if it's actually loop closure or not lol
else:
run small gating (local) + JCBB

landmark Revalidation: When an old landmark is successfully reobserved and confirmed through JCBB, it is reclassified as a new landmark, effectively resetting its age in the map, so now its in the N lifespan as a young again, so next time we check mathing, it will not check for thsi landmark until it has aged.

Age young landmarks that are still young. The landmarks that are already older than N dont need aging, just ignore them