\subsection{Integrated Data Association Framework for SLAM}
The complete DA pipeline combines local JCBB, global JCBB, descriptor based loop closure triggering, and landmark aging into a single unified framework. Each new scan follows the same sequence of operations, switching between local and global modes depending on whether a loop closure is likely.



\subsubsection{Processing Each New Scan}
For every incoming scan, the system first determines whether loop closure is plausible. This decision is made before any geometric DA step:
\begin{itemize}
    \item Perform descriptor similarity checks using a FLANN Matcher on all submaps older than the local horizon $N$. Young submaps are ignored to avoid trivially matching against very recent poses.
    \item A periodic fallback trigger (eks every 50 scans) is evaluated to guard against missed opportunities.
\end{itemize}
\noindent
If any descriptor match exceeds a predefined threshold or stands out strongly from all others, the system marks the scan as a loop closure candidate.



\subsubsection{Local vs Global DA Branch}
Once the trigger phase is complete, the DA mode is selected:
\begin{itemize}
    \item \textbf{If loop closure is suspected:} The system switches into global mode. A wide \emph{``global gating''} step is applied to the older map region, yielding a larger and more ambiguous candidate set. JCBB is then run using the full covariance structure $P_{xx},P_{xL},P_{LL}$ to verify the loop closure. This step is expensive, but it is executed rarely.
    \item \textbf{Otherwise (default case):} Local DA is performed. A tight \emph{``local gate''} is applied to the most recent $N$ submaps, producing only a few spatially separated candidates. JCBB then runs using only $P_{xx}$, which is cheap to extract and behaves similarly to an NN method when only a single candidate survives. This mode dominates runtime and ensures real-time performance.
\end{itemize}



\subsubsection{Landmark Aging}
Landmark management is handled via a simple aging mechanism:
\begin{itemize}
    \item Newly observed landmarks begin as ``young'' and remain inside the local window for $N$ scans.
    \item Each scan increments the age of all young landmarks.
    \item Once a landmark exceeds age $N$, it becomes an ``old'' landmark and joins the pool considered for loop closure triggering.
    \item Old landmarks do not need further aging, they remain in the global set indefinitely.
\end{itemize}



\subsubsection{Landmark Revalidation}
If an old landmark is successfully matched and passes JCBB during loop closure verification, it is revalidated:
\begin{itemize}
    \item Its age is reset to zero.
    \item It re-enters the local map as a young landmark.
\end{itemize}
\noindent
This prevents the system from repeatedly treating the same valid landmark as an old loop closure candidate and improves long term consistency.



\subsubsection{Summary of the Full Pipeline}
Putting everything together, each scan proceeds as follows:
\begin{enumerate}
    \item \textbf{Loop-closure trigger}: Evaluate descriptor similarity (FLANN) + periodic check on all submaps older than $N$.
    \item \textbf{If triggered}: Run global gating and full JCBB using $P_{xx},P_{xL},P_{LL}$ to verify loop closure.
    \item \textbf{If not triggered}: Run local gating and lightweight JCBB using only $P_{xx}$ on recent submaps.
    \item \textbf{Landmark aging}: Age all young landmarks, those exceeding $N$ join the loop closure candidate pool.
    \item \textbf{Landmark revalidation}: Any old landmark confirmed by JCBB becomes young again.
\end{enumerate}
\noindent
This framework ensures that the SLAM system stays fast during ordinary operation while still invoking full joint compatibility when it matters most, achieving both efficiency and robustness in data association.


