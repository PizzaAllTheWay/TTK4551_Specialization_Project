\subsection{Swath Processing}
\subsubsection{Raw Data and Waterfall Image Formation}
Each side scan sonar ping produces a 1D signal of backscatter intensity as a function of slant range, representing the acoustic energy reflected from the seafloor along the sonar beam path. As the vessel moves forward, consecutive pings are recorded sequentially and stacked along the track direction to form a 2D intensity image known as a \textit{``waterfall image''} or raw swath. In this representation, one axis corresponds to the ping index or along track distance, while the other corresponds to the across track slant range. The pixel intensity reflects the relative strength of the backscattered acoustic signal from the seafloor.
\\ \\
The waterfall image visualizes the seabed reflectivity pattern across successive pings but remains in slant range space, meaning that the vertical dimension represents the acoustic propagation distance rather than true horizontal ground distance. This introduces geometric distortion, particularly near the nadir, where the sonar beam intersects the seafloor at steep angles. As a result, the apparent scale of features varies with range, and the image does not yet represent the true seabed geometry.
\begin{figure}[H]
    \centering
    \includegraphics[width=1.0\linewidth]{Pictures/Local_Map_Generation/Swath_Processing/raw_waterfall_image.png}
    \caption{Example of a raw side scan sonar waterfall image showing backscatter intensity versus slant range for consecutive pings. The dark central band corresponds to the sonar blind zone.\textsuperscript{\cite{raw_waterfall_image}}}
    \label{fig:local-map-generation-raw-waterfall-image}
\end{figure}
\noindent
This raw slant range representation forms the foundation for subsequent swath processing. The following stages apply geometric and radiometric corrections to compensate for sensor motion, vehicle attitude, and acoustic signal decay, producing a consistent intensity map suitable for Cartesian projection and feature extraction. The implemented processing steps are based on the methodologies described in \cite{side_scan_sonar_master_thesis} and \cite{side_scan_sonar_paper}.



\subsubsection{Pitch and Roll Correction}
Variations in vessel attitude introduce geometric distortions in side scan sonar data. When the vessel experiences pitch or roll motion, the sonar beams deviate from their nominal orientation, causing the seafloor to appear tilted, curved, or stretched in the recorded swath. Without compensation, these effects lead to errors in the perceived seabed slope and displacement of features when the data is projected into Cartesian coordinates.
\\ \\
Pitch primarily influences the along track geometry by changing the effective depression angle of the sonar beam in the vertical plane. A positive pitch tilts the sonar forward, causing features to appear shifted ahead of their true positions. Roll, on the other hand, tilts the transducer laterally, altering the ground projection between the port and starboard channels. This asymmetry results in apparent lateral displacement or scaling differences of identical seabed features.
\begin{figure}[H]
    \centering
    \includegraphics[width=0.75\linewidth]{Pictures/Local_Map_Generation/Swath_Processing/non_zero_pitch.png}
    \caption{Starboard transducer and vessel geometry with non zero pitch. The figure shows how a positive pitch angle $\theta$ tilts the sonar beam forward relative to the horizontal plane, altering the beam incidence angle $\alpha_h$ and changing the apparent intersection point $p_s$ on the seafloor. $h_t$ denotes the vertical transducer height, while $h$ is the true vertical altitude of the vessels reference frame.\textsuperscript{\cite{side_scan_sonar_master_thesis}}}
    \label{fig:local-map-generation-pitch-correction}
\end{figure}
\begin{figure}[H]
    \centering
    \includegraphics[width=0.75\linewidth]{Pictures/Local_Map_Generation/Swath_Processing/non_zero_roll.png}
    \caption{Starboard transducer and vessel geometry with non zero roll. The roll angle $\phi$ tilts the transducer laterally, changing the apparent ground range $r_g$ and slant range $r_{fbr}$ for each channel. The mounting angle $\beta$ and the sonars vertical beamwidth $\alpha_v$ define the nominal beam orientation. A positive roll reduces the effective depression angle on the port side and increases it on the starboard side, causing geometric asymmetry between the channels.\textsuperscript{\cite{side_scan_sonar_master_thesis}}}
    \label{fig:local-map-generation-roll-correction}
\end{figure}
\noindent
To correct these distortions, each ping is compensated using the vessels estimated orientation obtained from the navigation state estimator fusing GNSS and IMU data. The estimated roll $\phi$ and pitch $\theta$ angles define the instantaneous orientation of the sonar beams relative to the horizontal plane and are used to transform the sonar measurements back into a consistent reference frame.

\paragraph{Explicit geometric correction} \mbox{}\\[0.5em] \noindent
The explicit geometric correction method estimates the true vertical altitude of each sonar transducer relative to the seabed using the first bottom return (FBR) in every ping. The FBR represents the first strong acoustic reflection received from the seafloor. Given the two way travel time $t_{fbr}$ to this reflection and the sound speed $c$ in water, the corresponding slant range is
\begin{equation}
    r_{fbr} = \frac{c \, t_{fbr}}{2}
    \label{eq:local-map-generation-fbr}
\end{equation}
which is the distance along the acoustic beam between the transducer and the point of reflection on the seafloor. This distance forms the hypotenuse of a right triangle, where the vertical component corresponds to the true transducer altitude above the seabed.
\\ \\
The vertical transducer height is obtained by projecting this slant range onto the vertical axis using the effective beam depression angle $\alpha$:
$$
    h_t = r_{fbr} \sin(\alpha)
$$
The total depression angle $\alpha$ depends on the fixed sonar mounting angle $\beta$ and the instantaneous pitch $\theta$ and roll $\phi$ of the vessel:
$$
    \alpha = \beta + \phi + \theta
$$
Since the vessel rolls about its longitudinal axis, the starboard and port transducers experience opposite roll directions relative to the horizontal plane. A positive roll angle $\phi$ tilts the starboard side downward (toward the seafloor) while lifting the port side upward (away from the seafloor). Therefore, the effective roll correction must be applied with opposite signs for each channel. Substituting these relationships yields:
\begin{equation}
    h_{t,stb} = \frac{c \, t_{fbr,stb}}{2} \sin(\beta + \phi + \theta) , \qquad
    h_{t,port} = \frac{c \, t_{fbr,port}}{2} \sin(\beta - \phi + \theta)
    \label{eq:local-map-generation-height-estimate}
\end{equation}
These heights represent the true vertical distance from each transducer to the seabed, accounting for both pitch and roll effects. Even for small roll angles, the difference between $h_{t,stb}$ and $h_{t,port}$ can be significant, especially in shallow water where small angular deviations cause noticeable geometric shifts.
\\ \\
Once the individual transducer heights are known, the corrected slant ranges to the first bottom return can be calculated as:
\begin{equation}
    r_{fbr,stb} = \frac{h_{t,stb}}{\sin(\beta + \phi + \frac{\alpha_v}{2})}, \qquad
    r_{fbr,port} = \frac{h_{t,port}}{\sin(\beta - \phi + \frac{\alpha_v}{2})}
    \label{eq:local-map-generation-slant-range-corrected}
\end{equation}
where $\alpha_v$ is the sonars vertical beamwidth. These relations account for roll induced asymmetry between the two channels, ensuring that each beam is projected correctly onto the seafloor. The resulting values are later used to identify the blind zone boundary and valid seafloor return bins.
\\ \\
Although the system used in this work does not include a Doppler Velocity Log (DVL) or altimeter, the transducer altitudes $h_{t,stb}$ and $h_{t,port}$ can still be estimated directly from the side scan sonar data using the FBR. The absence of external altitude sensors introduces additional uncertainty in these estimates, as the FBR detection depends on seabed reflectivity and acoustic noise. Nevertheless, the method provides sufficiently accurate geometric correction for surface vessels like MicroAmpere when combined with reliable pitch and roll estimates from the navigation state estimator.
\\ \\
By compensating each ping using the measured attitude angles and the estimated per channel transducer altitudes, the side scan sonar data are geometrically corrected for vessel motion, establishing a consistent base for blind zone removal and slant range correction.



\subsubsection{Blind Zone Removal}
After compensating for pitch and roll, the next processing stage identifies and removes the sonar blind zone. The blind zone corresponds to the near nadir region directly beneath the vessel where the sonar beam does not produce valid seabed returns. In this region, the acoustic pulse travels only through the water column before striking the seafloor, resulting in weak or noisy backscatter dominated by multipath reflections and surface interference.
\\ \\
The blind zone starts at the transducer position and extends to the first bottom return (FBR) for each ping. From the two-way travel time $t_{fbr}$, the raw slant range to the first seabed echo is given by Equation \ref{eq:local-map-generation-fbr}. However, this distance alone does not specify the true seabed intersection point, since vessel pitch and roll modify the effective beam depression angle. To correct for this, the per-channel transducer altitudes $h_{t,stb}$ and $h_{t,port}$ are estimated using Equation \ref{eq:local-map-generation-height-estimate}, which projects the measured slant range onto the vertical axis according to the current attitude.
\\ \\
Using these altitudes, the corrected slant ranges to the first valid seabed returns are computed from Equation \ref{eq:local-map-generation-slant-range-corrected}. The resulting values, $r_{fbr,stb}$ and $r_{fbr,port}$, represent the roll compensated limits beyond which the acoustic signal first reaches the seafloor. Each sonar beam contains a set of slant range samples denoted by $r_s$, and the first valid bottom return is located at $r_{fbr}$. When $r_s < r_{fbr}$, the signal originates from the water column and is discarded. When $r_s \ge r_{fbr}$, the return corresponds to the seafloor and is retained for mapping. This separation ensures that only physically meaningful data are passed to later correction stages.
\\ \\
In the raw waterfall image, the blind zone appears as a dark vertical band between the port and starboard channels. Its width scales with sonar altitude, higher transducer placement (larger $h_t$) produces a wider blind zone, while shallow operation results in a narrower central gap. The horizontal width can be derived geometrically from the sonar altitude and beam depression angle.
\\ \\
Consider one side of the sonar where the transducer height above the seabed is $h_t$ and the beam depression angle relative to the horizontal plane is $\beta$. The horizontal distance from the nadir (the point directly below the sonar) to the seabed intersection is denoted $x_b$. From the right triangle formed by the sonar, nadir, and seabed intersection,
$$
    \tan(\beta) = \frac{h_t}{x_b}
$$
Solving for $x_b$ gives
$$
    x_b = \frac{h_t}{\tan(\beta)}
$$
Since the blind zone covers both sides of the nadir, the total horizontal width becomes
$$
    w_b = 2x_b = 2 \frac{h_t}{\tan(\beta)} = 2 h_t \cot(\beta)
$$
where $h_t$ represents the mean of $h_{t,stb}$ and $h_{t,port}$, and $\beta$ is the fixed mounting depression angle. This simple geometric relation shows that increasing sonar altitude or reducing beam angle widens the blind zone proportionally.
\\ \\
Accurate detection and removal of the blind zone are essential for maintaining data quality. Unfiltered water column samples can introduce false intensities and distortions in later steps such as slant range correction and intensity normalization. After blind zone removal, each ping contains only valid seabed returns, providing a clean and geometrically consistent input for subsequent processing.
\begin{figure}[H]
    \centering
    \includegraphics[width=0.95\linewidth]{Pictures/Local_Map_Generation/Swath_Processing/blind_zone_example.png}
    \caption{Transverse view of a side scan sonar geometry showing the blind zone at nadir. The blind zone is the near-vertical region beneath the sonar where no valid seabed echoes are received, separating the port and starboard swaths.\textsuperscript{\cite{blind_zone_example}}}
    \label{fig:local-map-generation-blind-zone-example}
\end{figure}



\subsubsection{Slant Range Correction}
After blind zone removal, the remaining valid backscatter samples still represent distances along the sonar beam, known as \textit{``slant range''}. To correctly project these samples into a map of the seafloor, each measurement must be converted from slant range space to horizontal ground range coordinates. This step geometrically flattens the sonar data, forming the basis for accurate seabed mapping.
\\ \\
Although the corrected slant ranges $r_{fbr,stb}$ and $r_{fbr,port}$ obtained from Equation \ref{eq:local-map-generation-slant-range-corrected} account for vessel roll and pitch, they still describe distances along the acoustic path, not the horizontal ground distance. To produce a geometrically correct swath, each slant range sample is projected onto the seafloor plane assuming a locally flat seabed, which is a reasonable approximation for short range, high frequency side scan sonars.
\\ \\
Considering a right triangle formed by the sonar transducer, the seabed intersection point, and the vertical projection of the transducer onto the seabed, the geometric relation between the slant range $r_s$, sonar altitude $h_t$, and horizontal ground range $r_g$ is given by
$$
    r_s^2 = r_g^2 + h_t^2
$$
Here, $r_s$ represents the total slant range to a specific backscatter sample along the sonar beam, while $r_{fbr}$ denotes the first bottom return, the shortest valid slant range where the sonar beam first intersects the seabed. All samples with $r_s < r_{fbr}$ correspond to water column echoes and are discarded during blind zone removal, whereas samples with $r_s \ge r_{fbr}$ are considered valid seabed returns and are geometrically projected to ground range using the relation above. This conversion ensures that each valid sonar bin is positioned according to its true horizontal distance on the seafloor.
\\ \\
Rearranging for $r_g$ gives the horizontal projection formula:
$$
    r_g = \sqrt{r_s^2 - h_t^2}
$$
This transformation converts each intensity sample from sonar coordinates to its corresponding position on the seafloor. It effectively \textit{``flattens''} the swath geometry, ensuring that equal pixel distances in the final image correspond to equal ground distances on the seabed.
\\ \\
Without this correction, near nadir regions would appear compressed and far range regions stretched, leading to geometric distortion in the mapped imagery. Using the previously estimated per channel transducer heights $h_{t,stb}$ and $h_{t,port}$ from Equation \ref{eq:local-map-generation-height-estimate}, the transformation is applied separately to the port and starboard channels:
\begin{equation}
    r_{g,stb} = \sqrt{r_{s,stb}^2 - h_{t,stb}^2}, \qquad
    r_{g,port} = \sqrt{r_{s,port}^2 - h_{t,port}^2}
    \label{eq:local-map-generation-ground-range-projection}
\end{equation}
This step converts the 1D slant based scan line into a horizontal projection, enabling accurate along track stacking and Cartesian map generation in later stages.
\\ \\
The slant range correction relies on the assumption of a locally flat seafloor. While this assumption introduces small errors over complex terrain, it remains valid for the relatively short and narrow swaths typical of shallow water side scan systems such as MicroAmperes sonar. More advanced approaches, such as bathymetry aided correction or shadow based terrain estimation, can further refine this transformation by accounting for local seabed elevation variations. However, these methods are computationally intensive and typically require high resolution bathymetric data or complex image analysis. As such, they are not suitable for real-time operation on computationally constrained embedded systems or during online SLAM processing. For this work, the flat seabed assumption provides an optimal balance between accuracy and efficiency for onboard sonar mapping.
\begin{figure}[H]
    \centering
    \includegraphics[width=0.7\linewidth]{Pictures/Local_Map_Generation/Swath_Processing/slant_geometry.png}
    \caption{Geometric relation between slant range, transducer altitude, and horizontal ground range. The roll angle $\phi$ influences both the transducer height $h_t$ and the slant range $r_s$, affecting the final ground projection $r_g$.\textsuperscript{\cite{side_scan_sonar_paper}}}
    \label{fig:local-map-generation-slant-geometry}
\end{figure}



\subsubsection{Intensity Normalization}
The received sonar intensity naturally decays with range due to spherical spreading, absorption, and other propagation losses. These effects introduce unwanted attenuation in the measured signal, causing distant regions of the seabed to appear artificially darker. To counteract this, intensity normalization is applied to compensate for range dependent loss and ensure that the resulting image reflects true seabed backscatter rather than acoustic propagation effects.
\\ \\
Before normalization, the sonar signal has already undergone pitch and roll correction, blind zone removal, and slant range correction. The processed data now represent the measured acoustic backscatter intensity as a function of ground range for each ping. Each intensity sample is denoted as
$$
    I(i) = I(r_g(i))
$$
where $r_g(i)$ is the horizontal ground range obtained from the geometric projection (see Equation \ref{eq:local-map-generation-ground-range-projection}). Here, $r_g(i)$ defines the physical position of the $i$-th corresponding seafloor point, while $I(r_g(i))$ represents the measured acoustic intensity received from that $i$-th point.  
\\ \\
For simplicity, the notation $I(i)$ is used throughout the following sections to refer to the measured backscatter intensity associated with sample $i$, understood as the intensity evaluated at its ground projected location $r_g(i)$. This measured intensity still contains both the true seabed reflectivity and the effects of acoustic attenuation, beam pattern, and geometric spreading.
\\ \\
Intensity normalization can be performed either on a complete sonar image or on individual swaths. In this work, the correction is performed on individual swaths to enable real-time operation and maintain modularity in the processing chain. The method follows the framework described in \cite{side_scan_sonar_master_thesis}, where the measured sonar intensity $I(x,y)$ is modeled as a combination of two distinct components:
$$
    I(x, y) = R(x, y) \ast L(x, y)
$$
Here, $R(x,y)$ represents the \textit{``reflectance map''}, which describes the intrinsic acoustic backscatter of the seabed. It depends on physical factors such as surface roughness, material composition, and the local grazing angle. The second term, $L(x,y)$, represents the \textit{``illumination map''}, which models large scale variations in the measured signal caused by sonar specific effects like beam pattern, transmission loss, and range dependent attenuation. The operator $\ast$ denotes element wise multiplication since each pixels measured intensity is affected by both the true reflectivity of the seabed and the sonars illumination pattern at that location.
\\ \\
The goal of intensity normalization is to recover the true seabed reflectivity $R(x,y)$ by estimating and compensating for the illumination term $L(x,y)$. Once an estimate $\hat{L}(x,y)$ of the illumination map is obtained, the estimated reflectance map can be expressed as:
$$
    \hat{R}(x, y) = \frac{I(x, y)}{\hat{L}(x, y)} = R(x, y) \ast \frac{L(x, y)}{\hat{L}(x, y)}
$$
If the estimated illumination map $\hat{L}(x,y)$ closely approximates the true illumination $L(x,y)$, then $\hat{R}(x,y) \approx R(x,y)$, meaning the normalized image accurately represents the relative seabed reflectivity. This operation effectively removes large scale intensity variations caused by acoustic spreading, beam pattern effects, and medium absorption, ensuring that the remaining intensity variations primarily correspond to the physical properties of the seabed.
\\ \\
The illumination map $\hat{L}(x,y)$ varies slowly across the swath, representing low-frequency intensity trends rather than local seabed texture. It accounts for effects such as beam pattern variation, range dependent attenuation, and transducer gain. Several methods have been proposed to estimate this map. Filtering based approaches, such as mean or bilateral filtering, smooth the measured intensity to approximate the illumination field \cite{side_scan_sonar_master_thesis}. Physics based methods instead model the seabed as a Lambertian surface, using the relationship between grazing angle and backscatter strength to predict the illumination component \cite{side_scan_sonar_paper}.  
\\ \\
In this work, $\hat{L}(x,y)$ is estimated using smoothing cubic splines, which model the large scale intensity envelope as a smooth polynomial function fitted along each individual swath. This method effectively captures gradual illumination trends without overfitting to local texture variations. The spline is obtained by minimizing the cost function:
$$
    p \sum_{i=1}^{n} |I(i) - L(i)|^2 + (1 - p) \int_{1}^{n} |D^2L(t)|^2 \, dt
$$
where the first term measures the deviation from the observed intensity $I(i)$, and the second penalizes excessive curvature to ensure a smooth illumination field. The parameter $p$ controls the balance between accuracy and smoothness, while $D^2L$ denotes the second derivative of $L$, enforcing continuity and smooth transitions across the swath.  
\\ \\
After estimating $\hat{L}(i)$, the normalized reflectance profile is computed as:
\begin{equation}
    \hat{R}(i) = \frac{I(i)}{\hat{L}(i)}
    \label{eq:local-map-generation-reflectivity-normalization}
\end{equation}
This operation removes large scale illumination effects such as acoustic spreading and transducer gain variation, producing a consistent brightness profile across range while preserving the high frequency variations related to true seabed texture.  
\\ \\
The smoothing cubic spline approach provides a good compromise between computational efficiency and normalization accuracy. Unlike more complex illumination estimation techniques that require multi swath or physics based modeling, spline fitting can be computed quickly and independently for each swath, making it well suited for real-time embedded sonar processing. The smoothing parameter $p$ must be determined experimentally for the specific sonar setup, but it is typically small, on the order of magnitude less than $10^{-3}$. This ensures sufficient smoothing of large scale illumination variations while preserving fine scale seabed texture.
\\ \\
The result of the spline based normalization is a sonar image with a more uniform illumination profile, where artificial range dependent intensity variations are effectively suppressed. Figure \ref{fig:local-map-generation-intensity-normalization} illustrates this effect, the upper image shows the raw intensity with visible horizontal artifacts, while the lower image presents the normalized result with consistent brightness and clearer seabed texture across the swath.
\\ \\
This normalization method offers an effective balance between computational simplicity and correction accuracy. While more advanced reflectance estimation or global filtering approaches can produce smoother illumination compensation, they are typically too computationally heavy for real-time use. In contrast, the smoothing cubic spline method achieves sufficient accuracy for onboard sonar processing and real-time SLAM integration, making it a practical and efficient choice.
\begin{figure}[H]
    \centering
    \includegraphics[width=0.9\linewidth]{Pictures/Local_Map_Generation/Swath_Processing/intensity_normalized.png}
    \caption{The top image shows the blind zone removed sonar image with raw intensity constructed from the first 1000 swaths of the dataset. The bottom image shows the same image after smoothing cubic spline normalization, where artificial horizontal lines are almost invisible and overall illumination is more uniform.\textsuperscript{\cite{side_scan_sonar_master_thesis}}}
    \label{fig:local-map-generation-intensity-normalization}
\end{figure}



\subsubsection{Stacking and Reconstruction}
After each swath has been corrected for pitch, roll, slant range distortion, and intensity attenuation, the resulting data represents a consistent estimate of the seabed backscatter reflectivity in horizontal ground coordinates. To reconstruct a continuous seafloor image, the processed swaths are stacked sequentially in the temporal order of acquisition. This stacking corresponds directly to the vessels forward motion, where each new ping contributes a narrow strip of intensity data aligned along the along track axis.
\\ \\
During stacking, each swath is positioned according to the vessels estimated trajectory obtained from the navigation system, ensuring that positional drift and motion induced offsets are minimized. The along track resolution of the reconstructed image is determined by the ping rate and vessel speed, while the across track resolution depends on the sonars range sampling and slant to ground conversion.
\\ \\
The resulting 2D reconstruction forms an enhanced waterfall image where both axes correspond to physical ground distances and pixel intensity represents estimated seabed reflectivity. This corrected and normalized output marks the final stage of swath processing and serves as input for Cartesian projection and SLAM feature extraction.

