\subsection{Swath Processing}
This subsection covers the transformation of raw sonar returns into corrected intensity swaths. The raw data, typically received as a time-series of acoustic amplitudes per ping, undergoes preprocessing to remove noise, compensate for sensor orientation, and correct for acoustic geometry.  
\\ \\
The steps include pitch and roll correction using IMU data, blind zone removal near the nadir, and slant-range correction to flatten the seafloor projection. Intensity normalization is applied to compensate for range-dependent attenuation and transducer gain. The processed swaths are then stacked sequentially to form an intermediate waterfall image representing relative backscatter strength as a function of time and across-track distance.

