\subsection{Probabilistic Mapping}
\subsubsection{Mapping}
After swath processing, the sonar data is geometrically corrected, radiometrically normalized, and expressed in ground coordinates. The next stage is to reconstruct a consistent local map that spatially represents the acoustic backscatter intensity over the seafloor. This process, referred to as \textit{``probabilistic mapping''}, forms the core link between raw sonar measurements and the spatial representations required for SLAM based navigation and data association.
\\ \\
In conventional Cartesian mapping, each sonar return is assigned directly to a single grid cell using interpolation techniques such as k-Nearest Neighbor (kNN). While computationally efficient, this approach assumes that each sonar measurement corresponds to an exact reflection point on the seafloor, ignoring the finite width of the acoustic beam, the elliptical footprint, and range dependent uncertainty. As a result, such maps tend to appear artificially sharp yet physically inaccurate, particularly under conditions of noise, sparse coverage, or large angular incidence.
\\ \\
The probabilistic mapping framework addresses these limitations by representing each sonar beam as a spatial likelihood field rather than a deterministic ray. Instead of assigning a measurement to a single grid cell, its contribution is distributed among neighboring cells according to a physically motivated probability function derived from beam geometry, angular uncertainty, and measurement noise. This produces a continuous, uncertainty aware reflectivity map that better reflects the true acoustic interaction between the sonar and the seafloor surface.
\\ \\
Conceptually, the goal of the mapping stage is to transform sonar measurements into a locally consistent intensity field that preserves both geometric fidelity and radiometric realism. The resulting probabilistic map provides a robust foundation for feature extraction and SLAM, improving landmark repeatability across consecutive local maps and maintaining physical consistency even under dynamic vehicle motion and varying seabed conditions.



\subsubsection{Projection from Polar to Cartesian}
The projection from polar to Cartesian coordinates transforms each processed sonar swath from range bearing geometry into spatially consistent ground coordinates expressed in the NED frame. After slant range correction, every sonar sample corresponds to a point on the seafloor at a known horizontal ground range $r_g$ and bearing relative to the transducer. Each ping line represents an along-track position determined by the vehicles estimated trajectory, while the across track direction is defined by the projected ground range $r_g$ on either the port or starboard side.  
\\ \\
Given the vehicle position $\mathbf{p}_{b/O}^{n}$ and attitude quaternion $\mathbf{q}$ from the navigation state (See Equation \ref{eq:kinematics-motion-model-states}), each sonar measurement is transformed from the sonar frame to the NED frame through a complete chain of reference frames. The sonar is rigidly mounted on the vessel body with a fixed lever arm offset:
\begin{equation}
    \mathbf{r}_{s/b}^{b} =
    \begin{bmatrix}
        x_{s/b}^{b} \\ y_{s/b}^{b} \\ z_{s/b}^{b}
    \end{bmatrix}
    \label{eq:local-map-generation-sonar-mounting-lever-arm}
\end{equation}
and orientation $R_s^b$, defining the rotation from the sonar to the body frame. This rotation depends on the sonars physical installation and differs between the port and starboard transducers. Typically, the mounting angles include a yaw offset of $\pm 90^{\circ}$ relative to the vessels longitudinal axis and a pitch offset corresponding to the sonar depression angle $\beta$, while roll offset is often negligible under level installation \cite{side_scan_sonar_master_thesis,side_scan_sonar_paper}.  
\\ \\
After slant range correction, the ground projected location of each sonar return in the sonar frame is given by
$$
    \mathbf{p}_{g/s}^{s} =
    \begin{bmatrix}
        \pm r_g \\ 0 \\ 0
    \end{bmatrix}
$$
where $r_g$ is the horizontal ground range (see Equation \ref{eq:local-map-generation-ground-range-projection}), and the sign $\pm$ indicates whether the measurement originates from the port or starboard transducer. The notation $\mathbf{p}_{g/s}^{s}$ denotes the position of the seafloor point $g$ relative to the sonar origin $s$, expressed in sonar coordinates.  
\\ \\
The global position of this ground point in the NED frame is then computed through the full coordinate transformation:
\begin{equation}
    \mathbf{p}_{g/O}^{n} =
    R_b^n(\mathbf{q})
    \left( R_s^b \mathbf{p}_{g/s}^{s} + \mathbf{r}_{s/b}^{b} \right)
    + \mathbf{p}_{b/O}^{n}
    \label{eq:local-map-generation-polar-to-cartesian}
\end{equation}
where $R_b^n(\mathbf{q})$ represents the rotation matrix derived from the navigation quaternion, aligning the body frame with the NED frame.  
\\ \\
This transformation projects each sonar return from the transducers local reference frame into world referenced Cartesian coordinates, effectively merging multiple swaths into a coherent spatial map aligned with the vehicles estimated trajectory. It forms the geometric foundation for both classical Cartesian and probabilistic mapping methods, as both approaches rely on the same coordinate transformation to establish spatial consistency between sonar samples.


\subsubsection{kNN Interpolation and Grid Reconstruction}
Once all sonar samples have been transformed into global coordinates, they are used to populate a local Cartesian grid representing the seafloor reflectivity. Because sonar returns are irregularly distributed along the vehicles track and sparse in some regions, many grid cells may lack direct measurements. One of the classical approaches to address this problem is the k-Nearest Neighbor (kNN) interpolation method, which generates a continuous and dense reflectivity surface from irregularly spaced sonar samples. This method follows the interpolation framework presented in \cite{side_scan_sonar_master_thesis, kd_tree}.  
\\ \\
In this approach, a KD-tree is first constructed from all transformed sonar sample positions $\mathbf{p}_{g/O}^{n}$ to enable efficient nearest neighbor queries. For each grid cell centered at $(x_i, y_j)$, the $k$ closest sonar samples within a maximum search radius $r_{max}$ are identified. The Euclidean distance between a grid cell and its $k$-th nearest sample is given by:
$$
    d_k = \left\| 
    \mathbf{p}_{g/O}^{n}(k) -
    \begin{bmatrix} x_i \\ y_j \\ 0 \end{bmatrix}
    \right\|
$$
where $\mathbf{p}_{g/O}^{n}(k)$ denotes the global position of the $k$-th sonar sample in the NED frame.  
\\ \\
Each neighboring sample contributes to the grid cells estimated reflectivity according to its spatial proximity, weighted by an inverse distance weighting (IDW) function:
$$
    w_k = \frac{1}{((d_k)^p + \epsilon)}
$$
where $p$ is the weighting exponent (typically $1 \leq p \leq 2$) controlling the influence of nearby samples, and $\epsilon$ is a small constant to prevent division by zero. A higher $p$ value emphasizes nearby samples more strongly, producing sharper local contrast, while lower values yield smoother transitions.  
\\ \\
The final estimated reflectivity for a grid cell is obtained as a weighted mean of the $k$ nearest samples:
$$
    \hat{R}(x_i, y_j) =
    \frac{\sum_{k=1}^{K} w_k \, \hat{R}(k)}
         {\sum_{k=1}^{K} w_k}
$$
where $\hat{R}(k)$ denotes the normalized backscatter intensity of the $k$-th sonar sample obtained after radiometric correction (see Equation \ref{eq:local-map-generation-reflectivity-normalization}).  
\\ \\
Grid cells with no valid neighbors within the search radius $r_{max}$ remain empty or can optionally be filled by extrapolation. Outlier suppression may also be applied by rejecting distant or inconsistent neighbors to avoid spatial artifacts.  
\\ \\
The kNN interpolation method provides smooth and visually consistent maps while maintaining low computational cost, especially when implemented with KD-tree acceleration. It has complexity on the order of $\mathcal{O}(RC \log N)$, where $R$ and $C$ are grid dimensions and $N$ is the number of sonar samples. However, it assumes that each sonar return represents a precise reflection point, neglecting factors such as beam footprint, angular uncertainty, and range dependent spreading effects.  
\\ \\
While effective as a geometric baseline, this deterministic interpolation approach lacks the physical realism required for uncertainty aware SLAM and feature extraction. Therefore, it is presented here primarily as a reference method, the final system instead employs a probabilistic formulation that models sonar beam uncertainty and distributes reflectivity using likelihood based mapping.
\begin{figure}[H]
    \centering
    \includegraphics[width=0.9\linewidth]{Pictures/Local_Map_Generation/Probabilistic_Mapping/2D_KD_tree.png}
    \caption{Illustration of a 2D KD-tree used for k-Nearest Neighbor search. (a) shows the spatial partitioning of sample points in the $(x,y)$ plane, and (b) shows the corresponding binary tree structure where each node alternates between splitting along the $x$ and $y$ axes. This hierarchical structure enables efficient $k$-nearest neighbor queries for interpolating sonar samples in the Cartesian grid.\textsuperscript{\cite{kd_tree}}}
    \label{fig:local-map-generation-2d-kd-tree}
\end{figure}



\subsubsection{Probabilistic Interpolation and Grid Reconstruction}
The probabilistic interpolation and grid reconstruction method represents the physically consistent mapping strategy used in this work. Unlike deterministic interpolation techniques such as kNN, which assume each sonar return originates from a single point, the probabilistic approach models each sonar beam as a spatial likelihood field. Each measurement contributes not only to the grid cell it directly intersects but also to its neighbors, with weights derived from the sonars beam geometry, angular uncertainty, and range dependent error characteristics.  

\paragraph{Input} \mbox{}\\[0.5em] \noindent
The probabilistic mapping process begins with the output of the geometric projection step (Equation \ref{eq:local-map-generation-polar-to-cartesian}). After this transformation, every sonar return is represented in global coordinates as $\mathbf{p}_{g/O}^{n}(i)$ and associated with a normalized intensity value $\hat{R}(i)$ obtained from the reflectivity normalization stage (see Equation \ref{eq:local-map-generation-reflectivity-normalization}).  
\\ \\
Here, $\mathbf{p}_{g/O}^{n}(i)$ denotes the position of the $i$-th seafloor reflection point expressed in the NED frame, while $\hat{R}(i)$ represents its corresponding measured acoustic backscatter intensity. The term $\mathbf{p}$ therefore describes a \textit{``spatial location''} in 3D space, whereas $\hat{R}$ refers to a \textit{``measured physical quantity''} (reflectivity) at that location. 
\\ \\ 
Together, they form the set of sonar observations:
$$
    \mathcal{S} = \left\{ \mathbf{p}_{g/O}^{n}(i), \, \hat{R}(i) \right\}_{i=1}^{N_s}
$$
where $N_s$ is the total number of processed sonar samples across all acquired swaths.  
Each processed swath (indexed by $m$) corresponds to one sonar ping sequence, and within each swath, $i$ indexes the individual intensity samples belonging to that specific ping.
\\ \\  
Thus, the total number of processed samples is given by:
$$
    N_s = \sum_{m=1}^{N} N_m
$$
where $N$ is the total number of processed swaths used for local map generation, and $N_m$ is the number of valid sonar samples within the $m$-th swath.  
\\ \\
In an idealized case with a constant number of valid samples per swath, this could be simplified as $N_s = N \cdot N_m$. However, in practice, each $m$-th processed swath or ping contains a different number of valid samples due to blind zone removal, attitude correction, and range dependent filtering. Therefore, every swath must be processed individually before aggregating all valid samples into a coherent probabilistic reflectivity map.
\\ \\
These samples together form the complete input dataset for the probabilistic grid reconstruction stage.

\paragraph{Goal and Grid Definition} \mbox{}\\[0.5em] \noindent
The objective of the mapping step is to populate a 2D Cartesian grid representing the local seafloor reflectivity distribution. This grid is defined over world coordinates $q=(x, y)$ and stores the estimated reflectivity $\hat{R}_{prob\,map}(x, y)=\hat{R}_{prob\,map}(q)=V(q)$ for each cell.  
\\ \\
The center of every grid cell is represented by the vector:
$$
    \mathbf{p}_{q/O}^{n} =
    \begin{bmatrix}
        x_q \\ y_q \\ 0
    \end{bmatrix}
$$
which defines the spatial position of grid cell $q$ in the NED frame.  
\\ \\
The end goal is to fill this grid with physically consistent reflectivity values that reflect how the sonar beams illuminate the seafloor over time. 
\\ \\
In deterministic approaches such as kNN interpolation, each sonar return contributes only to the cell that lies closest to it. In contrast, the probabilistic approach considers that a sonar beam has a finite spread and that each measurement may influence several neighboring grid cells. This reduces artificial \textit{``dark gaps''} between samples and produces a smoother, more realistic seafloor representation.  
\\ \\
Conceptually, each sonar return $\mathbf{p}_{g/O}^{n}(k)$ contributes to the map by distributing its intensity $\hat{R}(k)$ across nearby grid cells according to the physical beam model. However, evaluating every sonar point against every grid cell is computationally expensive. Naively, this requires $\mathcal{O}(N_s N_g)$ operations, where $N_s$ is the number of processed swath sonar samples and $N_g$ is the number of grid cells. Later, an accelerated method proposed by \cite{side_scan_sonar_master_thesis} will be discussed, that reduces this cost by ignoring cells too far or too oblique to be realistically illuminated.

\paragraph{Beam Geometry and Observation Angles} \mbox{}\\[0.5em] \noindent
The sonar transducer emits acoustic beams characterized by a finite horizontal opening angle $\alpha_h$, which defines the angular width of the acoustic footprint on the seafloor. This parameter is determined by the transducer design and is illustrated in Figures \ref{fig:local-map-generation-pitch-correction} and \ref{fig:local-map-generation-azimuth-geometry}.  
To determine whether a given grid cell $q$ could be illuminated by a specific processed swath sample $m$ (that is, a particular sonar ping), the corresponding \textit{``observation angle''} $\theta_{obs}$ is computed in the sonar frame.
\\ \\
Each sonar ping, indexed by $m$, corresponds to one transmitted acoustic pulse (swath) that produces a sequence of $N_m$ valid intensity samples $\hat{R}(i)$ distributed along range. The beam geometry and observation angles are evaluated independently for each ping to determine which grid cells fall within the effective acoustic footprint of that single transmission.
\\ \\
For each grid cell position $\mathbf{p}_{q/O}^{n}$ and sonar ping $m$, the relative position of the grid cell with respect to the sonar origin is computed as:
$$
    \mathbf{r}_{q/s}^{s} = (R_s^b)^\top (R_b^n)^\top 
    \left(
        \mathbf{p}_{q/O}^{n}
        - \mathbf{p}_{b/O}^{n}
        - R_b^n(\mathbf{q}) \mathbf{r}_{s/b}^{b}
    \right)
$$
where $\mathbf{p}_{b/O}^{n}$ is the vehicle position in the NED frame, obtained from the state estimator.$R_b^n(\mathbf{q})$ is the body to NED rotation matrix derived from the navigation state quaternion, also provided by the state estimator. $\mathbf{r}_{s/b}^{b}$ is the sonar lever arm offset relative to the body frame, a static, known constant defined by the sonars mounting position relative to the vehicles center of gravity (COG) (see Equation \ref{eq:local-map-generation-sonar-mounting-lever-arm}). Finally $R_s^b$ is the fixed rotation matrix from the sonar to the body frame, determined by the sonars installation orientation (see below Equation \ref{eq:local-map-generation-sonar-mounting-lever-arm}).
\\ \\
This expression converts the position of grid cell $q$ from world coordinates to the sonar frame at the time of ping $m$.  
The resulting vector $\mathbf{r}_{q/s}^{s} = [r_{x}^{s}, r_{y}^{s}, r_{z}^{s}]^\top$ represents the position of cell $q$ relative to the sonar.  
\\ \\
From this vector, the observation (bearing) angle is obtained by:
\begin{equation}
    \theta_{obs} = \arctan2(r_{y}^{s}, \, r_{x}^{s})
    \label{eq:local-map-generation-observation-angle}
\end{equation}
This angle measures the azimuthal direction from the sonar to the grid cell relative to the beams acoustic axis. If $\theta_{obs} = 0$, the grid cell $q$ lies exactly on the beam centerline. As $|\theta_{obs}|$ increases, the following cells lie further and further to the side, and if the deviation exceeds half the beamwidth, the cell is no longer illuminated:
$$
    |\theta_{obs}| < \frac{\alpha_h}{2}
$$
Intuitively, $\alpha_h$ defines how wide the sonar \textit{``sees''} horizontally, while $\theta_{obs}$ tells us whether a particular cell lies inside that visible region.
\begin{figure}[H]
    \centering
    \includegraphics[width=0.85\linewidth]{Pictures/Local_Map_Generation/Probabilistic_Mapping/Azimuth_Beam_Geometry.png}
    \caption{Illustration of the sonar beams horizontal geometry showing the acoustic axis, horizontal beamwidth $\alpha_h$, and the observation angle $\theta_{obs}$ to a grid cell $q$. The cell lies within the illuminated footprint if $|\theta_{obs}| < \alpha_h/2$.\textsuperscript{\cite{side_scan_sonar_master_thesis}}}
    \label{fig:local-map-generation-azimuth-geometry}
\end{figure}
\noindent
The way to interpret this process is as follows. At the beginning of a swath acquisition (ping sequence $m$), only the grid cells closest to the sonar and within its current field of view are illuminated, the remaining distant cells are outside the observation cone and can be ignored for that specific swath.  
\\ \\
As the vehicle moves and new pings are emitted, the illuminated region gradually shifts across the grid.  
Cells that were visible in earlier pings may no longer fall within the beams angular range, while previously unseen cells begin to be illuminated as the sonar advances along its trajectory.  
\\ \\
In essence, the probabilistic mapping process resembles a simplified, physically constrained form of ray tracing, where each sonar beam acts as a moving illumination cone that progressively scans the seafloor. However, unlike computer graphics lighting models, this illumination is governed by physical sonar geometry and modeled using Gaussian probability distributions to represent beam divergence and angular uncertainty, as will be detailed in the next section.

\paragraph{Probabilistic Beam Model} \mbox{}\\[0.5em] \noindent
In deterministic interpolation methods such as kNN, only the nearest grid cells are updated, resulting in sharp intensity boundaries and artificial gaps. In contrast, the probabilistic beam model treats each sonar beam as a continuous angular likelihood field that spreads its energy according to a Gaussian distribution centered on the beam axis:
$$
    p(\theta_{obs}) = \frac{2}{\alpha_h \sqrt{2\pi}} 
    \exp\left(-\frac{2\theta_{obs}^2}{\alpha_h^2}\right)
$$
This function represents the relative contribution of a sonar beam at an angular deviation $\theta_{obs}$ from its acoustic centerline. The intensity is strongest near $\theta_{obs} = 0$ and gradually decreases toward the beam edges ($|\theta_{obs}| \approx \alpha_h / 2$). (See Equation \ref{eq:local-map-generation-observation-angle})  
\\ \\
For each sonar ping $m$, the beam spans a finite angular region defined by its horizontal beamwidth $\alpha_h$.  
The nominal angular limits for that ping are:
$$
    \theta_{obs,\min}^m = -\frac{\alpha_h}{2}, \qquad 
    \theta_{obs,\max}^m = +\frac{\alpha_h}{2}
$$
However, the actual observation angle $\theta_{obs}(q)$ for each grid cell $q$ depends on its position relative to the sonar ping $m$, computed earlier. (See Equation \ref{eq:local-map-generation-observation-angle})
\\ \\
Each cell subtends a small angular width $\Delta \theta_q$ as seen from the sonar. The probability that cell $q$ is illuminated by ping $m$ is therefore found by integrating the beams angular intensity profile over the angular span corresponding to that specific cell:
$$
    P_m(q) = \int_{\theta_{obs}(q) - \frac{\Delta \theta_q}{2}}^{\theta_{obs}(q) + \frac{\Delta \theta_q}{2}} p(\theta_{obs}) \, d\theta_{obs}
$$
Here, the integration limits differ for every grid cell because each has a unique observation angle $\theta_{obs}(q)$.  
This integration effectively converts the continuous beam intensity function into a discrete per cell probability, summing the energy that falls within the angular extent of that cell.
\\ \\
In practice, the grid cells are small enough that this integral can be accurately approximated by evaluating the Gaussian at the cell center:
$$
    P_m(q) \approx p\big(\theta_{obs}(q)\big)
$$
Before performing this computation, most grid cells can be safely excluded to reduce computational cost. For each sonar ping $m$, only the subset of grid cells that can physically be illuminated are considered and those within the valid sonar range window
$$
    |\theta_{obs}(q)| < \frac{\alpha_h}{2}
$$
All other cells are ignored for that ping, as they receive negligible acoustic energy.
\\ \\
Algorithmically, this means that for each sonar ping $m$, a probability grid $P_m(q)$ is initialized to zero and updated only for the grid cells that lie within the beams angular and range limits. Cells outside this region are skipped entirely. As the vehicle moves and subsequent pings are processed, each $P_m(q)$ contributes to a different local region of the map. Over time, these overlapping illuminated regions merge into a coherent probabilistic reflectivity map of the seafloor.
\\ \\
Finally, the practical formulation used to evaluate $P_m(q)$ in this work is given by:
\begin{equation}
    P_m(q) =
    \begin{cases}
        \displaystyle
        \frac{2}{\alpha_h \sqrt{2\pi}} 
        \exp\!\left(-\frac{2\,\theta_{obs}(q)^2}{\alpha_h^2}\right),
        & \text{if } |\theta_{obs}(q)| < \frac{\alpha_h}{2} \\[1.0em]
        0, & \text{otherwise}
    \end{cases}
    \label{eq:local-map-generation-probability-weight-individual-swaths}
\end{equation}
This expression directly links each grid cells spatial position to its illumination probability for a given sonar ping $m$, ensuring that only geometrically valid and acoustically illuminated cells contribute to the probabilistic map. The observation angle $\theta_{obs}(q)$ used in this equation is computed from the geometric relationship previously defined in Equation \ref{eq:local-map-generation-observation-angle}. It describes how far the grid cell $q$ lies laterally from the sonars acoustic axis, based on its position relative to the sonar in the local sonar frame.  
\\ \\
Before performing any probability computation, it is first verified whether the grid cell $q$ can realistically be illuminated by the current $m$-th ping. This pre-filtering step avoids unnecessary computation by immediately rejecting cells that fall outside the sonars effective angular footprint or range limits. Only the subset of grid cells satisfying $|\theta_{obs}(q)| < \alpha_h / 2$ and located within the valid range interval are processed further using Equation \ref{eq:local-map-generation-probability-weight-individual-swaths}. Cells that fail these conditions receive zero contribution for that ping and are skipped entirely.  
\\ \\
For each processed sonar swath (ping) $m$, the algorithm constructs a temporary probabilistic grid $P_m(q)$ that represents how strongly that single ping contributes to the surrounding grid cells. The observation angle for each cell is calculated using Equation \ref{eq:local-map-generation-observation-angle}, and the illumination probability is then evaluated through the Gaussian beam weighting in Equation \ref{eq:local-map-generation-probability-weight-individual-swaths}. This procedure is repeated independently for every ping $m = 1, 2, \ldots, N$, where $N$ is the total number of processed swaths used for local map generation.  
\\ \\
The final probabilistic reflectivity map is then obtained by fusing all individual per ping contributions according to the complementary probability rule:
\begin{equation}
    P(q) = \sum_{m=1}^{N} P_m(q)
    \label{eq:local-map-generation-probability-weight-total}
\end{equation}
This formulation ensures that repeated illumination of the same grid cell across multiple pings increases its overall probability of observation in a physically consistent way, without exceeding unity. Each sonar ping therefore contributes a localized Gaussian weighted illumination pattern to the grid, and as the vehicle moves, these overlapping contributions accumulate into a continuous probabilistic map of the seafloor reflectivity.
\begin{figure}[H]
    \centering
    \includegraphics[width=0.85\linewidth]{Pictures/Local_Map_Generation/Probabilistic_Mapping/P(q)_Map.png}
    \caption{Visualization of the resulting probability map $P(q)$ in normalized form for better visualization. The color bar represents the illumination probability of each grid cell $q$, where darker color indicate higher likelihood of observation. Each visible line corresponds to the contribution from a single pings $P_m(q)$ field. When all $P_m(q)$ distributions are combined using Equation \ref{eq:local-map-generation-probability-weight-total}, the result is a complete probabilistic map showing the accumulated illumination coverage of the seafloor.\textsuperscript{\cite{side_scan_sonar_master_thesis}}}
    \label{fig:local-map-generation-probability-map}
\end{figure}

\paragraph{Interpolation from Waterfall to Grid Representation} \mbox{}\\[0.5em] \noindent
After swath processing, each ping $m$ provides a 1D sequence of normalized seabed reflectivity values $\hat{R}_m(i)$ distributed along the horizontal ground range $r_g(i)$ (see Equation \ref{eq:local-map-generation-reflectivity-normalization}). Together, these values form the so called \textit{``waterfall image''}, which represents intensity as a function of range and ping index. Although this view is geometrically corrected and normalized, it still lies in the sonars acquisition geometry and not in the Cartesian grid used for mapping.  
\\ \\
Before the probabilistic map $P(q)$ and its reflectivity field $V(q)$ can be formed, the 1D waterfall data of each swath must be projected into the 2D spatial grid shared by the map representation. This requires interpolating the discrete reflectivity samples $\hat{R}_m(i)$, defined along $r_g(i)$, into corresponding grid cell positions $q=(x_q,y_q)$ that lie within the sonar footprint of that ping.
\\ \\
Each grid cell $q$ has fixed Cartesian coordinates $q=(x_q, y_q)$ in the NED frame and four corner points $\mathbf{p}_q(i)$, where $i=0,1,2,3$. The sonar position at the time of ping $m$ is given by
$$
    \mathbf{p}_{s/O}^{n}(m) = R_b^n(\mathbf{q})\mathbf{r}_{s/b}^{b} + \mathbf{p}_{b/O}^{n}
$$
where $\mathbf{p}_{b/O}^{n}$ and $\mathbf{q}$ is the vessel position and attitude obtained from the state estimator at the instant of ping $m$, and $\mathbf{r}_{s/b}^{b}$ is the fixed lever arm offset of the sonar relative to the vessel body frame. 
\\ \\ 
Using these quantities, the slant range distance from the sonar $\mathbf{p}_{s/O}^{n}(m)$ at ping $m$ to each corner of grid cell $\mathbf{p}_q(i)$ is calculated as
$$
    r_{s,i}^m = \left\| \mathbf{p}_q(i) - \mathbf{p}_{s/O}^{n}(m) \right\|
$$
Each $r_{s,i}^m$ represents the acoustic distance from the sonar transducer to the $i$-th corner of cell $q$, corresponding to a position along the measured 1D intensity profile $\hat{R}_m(i)$. Later in the acceleration discussion, methods for efficiently pruning grid cells that are outside the active sonar footprint will be presented, ensuring that only a small subset of relevant cells is processed for each ping $m$.
\\ \\
Because the measured intensity samples $\hat{R}_m(i)$ are defined only at discrete range intervals of size $\delta_s$ (See Figure \ref{fig:local-map-generation-azimuth-geometry}), interpolation is required to estimate the intensity value at an arbitrary slant range $r_{s,i}^m$. For this purpose, linear interpolation is applied directly between the two nearest measured samples along the 1D profile.  
\\ \\
The interpolation weights are computed using the normalized ratio between the desired range and the sampling interval, defined as
$$
    w_1 = \frac{r_{s,i}^m}{\delta_s} - 
          \left\lfloor \frac{r_{s,i}^m}{\delta_s} \right\rfloor, 
    \qquad
    w_2 = 1 - w_1
$$
Here, the floor operator $\lfloor \cdot \rfloor$ extracts the integer part of the normalized range $\frac{r_{s,i}^m}{\delta_s}$, giving the index of the nearest lower sampled range in the processed swath of the $m$-th ping. The fractional part represents how far the desired range lies between two consecutive samples in this swath. If $r_{s,i}^m$ is close to the lower sample, $w_1$ becomes small and $w_2$ dominates, if it is near the upper sample, $w_1$ increases while $w_2$ decreases. This provides a smooth linear weighting between the two closest intensity samples of the processed swath. These weights can be seen as a simple yet intelligent variant of kNN interpolation, where the two nearest range samples are blended proportionally to their distance from the target point.
\\ \\
The corresponding interpolated reflectivity value is then given by
$$
    \hat{R}_m(r_{s,i}^m) =
    w_1 \, \hat{R}_m\!\left(
        \left\lfloor \frac{r_{s,i}^m}{\delta_s} \right\rfloor
    \right)
    + 
    w_2 \, \hat{R}_m\!\left(
        \left\lceil \frac{r_{s,i}^m}{\delta_s} \right\rceil
    \right)
$$
Here, both $\lfloor \cdot \rfloor$ and $\lceil \cdot \rceil$ represent discrete sample indices in the processed swath of ping $m$. Since the measured intensities $\hat{R}_m(i)$ are sampled at uniform range intervals $\delta_s$, the normalized value $\frac{r_{s,i}^m}{\delta_s}$ effectively gives the \textit{``fractional sample index''} corresponding to the slant range $r_{s,i}^m$. The floor and ceiling operators then select the two closest measured samples, the \textit{``neighbors''} of that point, while the weights $w_1$ and $w_2$ determine how much each contributes.  
\\ \\
These weights linearly interpolate the measured intensities between the two nearest samples in the processed swath, ensuring smooth transitions between adjacent ranges. Averaging the interpolated corner values yields the mean estimated reflectivity $V_m(q)$ for each grid cell as observed by ping $m$:
\begin{equation}
    V_m(q) = \frac{1}{4}\sum_{i=0}^{3}\hat{R}_m(r_{s,i}^m)
    =
    \frac{1}{4}\sum_{i=0}^{3}
    \Bigg[
        w_1 \, \hat{R}_m\!\left(
            \left\lfloor \frac{r_{s,i}^m}{\delta_s} \right\rfloor
        \right)
        +
        w_2 \, \hat{R}_m\!\left(
            \left\lceil \frac{r_{s,i}^m}{\delta_s} \right\rceil
        \right)
    \Bigg]
    \label{eq:local-map-generation-interpolated-intensity}
\end{equation}
Intuitively, this step converts the 1D processed swath, a line of measured reflectivity samples along the sonar range, into a 2D spatial representation aligned with the mapping grid. Each grid cell receives a reflectivity estimate based on how its corners relate to the measured intensity profile of the current ping, effectively \textit{``painting''} that pings information onto the map with correct geometry and resolution.  
\\ \\
This interpolation acts as a geometric resampling process. It transfers each pings reflectivity data from the waterfall domain into the Cartesian grid used by the probabilistic mapping framework. Once this interpolation is complete, the resulting per ping reflectivity field $V_m(q)$ can be directly fused with the corresponding probability weights $P_m(q)$ to construct the global occupancy and reflectivity map in a spatially consistent manner.

\paragraph{Reflectivity Fusion} \mbox{}\\[0.5em] \noindent 
After each ping $m$ has been interpolated into the grid as the per ping reflectivity field $V_m(q)$ (see Equation \ref{eq:local-map-generation-interpolated-intensity}), and weighted by its associated probability distribution $P_m(q)$ (see Equation \ref{eq:local-map-generation-probability-weight-individual-swaths}), all overlapping contributions are fused together into a single reflectivity estimate for each grid cell. Producing a weighted sum of all pings contributing to the same cell:
\begin{equation}
    V(q) = \sum_{m=1}^{N} P_m(q) \, V_m(q)
    \label{eq:local-map-generation-reflectivity-fusion}
\end{equation}
This formulation ensures that overlapping beams merge smoothly, while measurements with higher probability (ie, stronger and better aligned sonar returns) have a greater influence on the resulting grid intensity. Physically, this represents a spatial blending of all local swath reflectivity estimates based on their observation likelihoods.

\paragraph{Probability Map Normalization} \mbox{}\\[0.5em] \noindent 
Once the reflectivity fusion in Equation \ref{eq:local-map-generation-reflectivity-fusion} has been performed, the final reflectivity map is normalized by the total accumulated observation probability $P(q)$ from Equation \ref{eq:local-map-generation-probability-weight-total}. The normalized reflectivity value for each grid cell is given by:
\begin{equation}
    M(q) = \frac{V(q)}{P(q)}
    \label{eq:local-map-generation-probability-normalization}
\end{equation}
This step compensates for varying observation density across the map, ensuring consistent intensity scaling even in regions observed by multiple overlapping pings. As a result, $M(q)$ represents the final, fully fused, and normalized local reflectivity map, a spatially consistent depiction of the seafloor surface ready for SLAM and feature extraction.

\paragraph{Post processing and Gap Filling} \mbox{}\\[0.5em] \noindent 
After map fusion, small unobserved gaps may remain between neighboring swaths. These are filled using a kNN extrapolation, where each empty cell is assigned the mean intensity of its $k$ nearest observed neighbors within a defined distance threshold. To optimize computation, a flood fill method is used to identify only those cells within the active sonar footprint $Q_m$, avoiding unnecessary evaluation of the full grid.

\paragraph{Algorithm Evaluation} \mbox{}\\[0.5em] \noindent
The probabilistic interpolation and grid reconstruction algorithm provides a physically accurate and uncertainty aware mapping framework, effectively modeling the sonar beams angular spread and measurement noise. However, the complete formulation is computationally heavy, requiring the evaluation of thousands of potential grid cells per ping and multiple Gaussian weighting operations.  
\\ \\
As a result, the full probabilistic approach is unsuitable for real-time use. Therefore, in the next chapter, the optimized multi stage pruning and caching algorithm from \cite{side_scan_sonar_master_thesis} and \cite{side_scan_sonar_paper} is introduced. This improved implementation maintains the same probabilistic consistency while drastically reducing the computational load, enabling real-time operation on microAmpere ASV without sacrificing map quality.
\begin{figure}[H]
    \centering
    \includegraphics[width=0.85\linewidth]{Pictures/Local_Map_Generation/Probabilistic_Mapping/Probabilistic_Map.png}
    \caption{Final probabilistic reflectivity map after fusion and gap filling.\textsuperscript{\cite{side_scan_sonar_master_thesis}}}
    \label{fig:local-map-generation-probabilistic-map}
\end{figure}



\subsubsection{Optimized Probabilistic Interpolation and Grid Reconstruction}
\paragraph{Computation and Acceleration} \mbox{}\\[0.5em] \noindent 
A direct implementation of the probabilistic map generation would require comparing every sonar sample with every grid cell, leading to a computational complexity of $\mathcal{O}(N_s N_g)$, where $N_s$ is the number of sonar samples and $N_g$ the total number of grid cells. Such a brute force approach is impractical for real-time operation. To achieve efficient execution, the algorithm applies a multi-stage pruning and caching strategy that incrementally narrows down the active grid subset for each sonar ping $m$ before interpolation and fusion are performed.
\\ \\
The pruning process is organized into 3 hierarchical stages, each discarding unnecessary grid cells based on progressively more selective criteria.

\paragraph{Stage 1: Range Based Pruning} \mbox{}\\[0.5em] \noindent
The first and cheapest filtering step limits grid cells based on their slant range relative to the sonar position $\mathbf{p}_{s/O}^{n}(m)$. Since the sonar pose and the range bounds $r_{\min}$ and $r_{\max}$ are constant for each ping $m$, they are computed once and reused (See Equation \ref{eq:local-map-generation-ground-range-projection} and Figure \ref{fig:local-map-generation-azimuth-geometry}). All cells are evaluated using the Euclidean distance
$$
    r_q = \| \mathbf{p}_q - \mathbf{p}_{s/O}^{n}(m) \|
$$
and only those satisfying $r_{\min} < r_q < r_{\max}$ are kept. This operation quickly eliminates most of the map outside the sonar footprint, typically reducing the grid size by over an order of magnitude.

\paragraph{Stage 2: Angular Pruning} \mbox{}\\[0.5em] \noindent
The remaining grid cells are then pruned by angular visibility using the observation angle $\theta_{obs}(q)$ (see Equation \ref{eq:local-map-generation-observation-angle} and Figure \ref{fig:local-map-generation-azimuth-geometry}). The precomputed constants from the previous stage are reused to help calculate $\theta_{obs}(q)$ efficiently. Cells outside the effective horizontal beam aperture are removed:
$$
    |\theta_{obs}(q)| < \frac{\alpha_h}{2}
$$
This second stage eliminates cells lying outside the sonars field of view, further shrinking the active grid subset.

\paragraph{Stage 3: Probability Based Pruning} \mbox{}\\[0.5em] \noindent
For the remaining candidates, the illumination probability $P_m(q)$ is evaluated using the beam model in Equation \ref{eq:local-map-generation-probability-weight-individual-swaths}. The precomputed observation angle $\theta_{obs}(q)$ from the previous stage are reused to help calculate $P_m(q)$ efficiently. Cells with negligible contribution are discarded using a small predefined threshold $\epsilon$, yielding the final refined set:
$$
    Q_m = \left\{
        q \in Q \; \bigg| \;
        r_{\min} < \| \mathbf{p}_q - \mathbf{p}_{s/O}^{n}(m) \| < r_{\max},
        \;
        |\theta_{obs}(q)| < \frac{\alpha_h}{2},
        \;
        P_m(q) \geq \epsilon
    \right\}
$$
This final filtering step removes cells whose probability of illumination is too low to affect the map, reducing the computational domain by several additional orders of magnitude. In practice, the number of candidate cells typically decreases from millions in the full grid, to tens of thousands after range pruning, to a few thousand after angular pruning, and finally to a few hundred high probability cells after the last stage. These cells form the final set $Q_m$, which is then used for interpolation and probabilistic fusion.

\paragraph{Optimized Equations} \mbox{}\\[0.5em] \noindent
The modified, computationally efficient version of the complete probabilistic map generation process can then be expressed as:
\begin{equation}
    Q_m = \left\{
        q \in Q \; \bigg| \;
        r_{\min} < \| \mathbf{p}_q - \mathbf{p}_{s/O}^{n}(m) \| < r_{\max},
        \;
        |\theta_{obs}(q)| < \frac{\alpha_h}{2},
        \;
        P_m(q) \geq \epsilon
    \right\}
    \label{eq:local-map-generation-multistage-pruning-fast} \\[0.5em]
\end{equation}
\begin{equation}
    V_m(q) = \frac{1}{4} \sum_{i=0}^{3} 
        \Bigg[
        w_1 \, \hat{R}_m\!\left(
            \left\lfloor \frac{r_{s,i}^m}{\delta_s} \right\rfloor
        \right)
        +
        w_2 \, \hat{R}_m\!\left(
            \left\lceil \frac{r_{s,i}^m}{\delta_s} \right\rceil
        \right)
    \Bigg]
    \label{eq:local-map-generation-interpolation-fast} \\[0.5em]
\end{equation}
\begin{equation}
    V(q) = 
    \sum_{m \in S}
    \begin{cases}
        0, & q \notin Q_m \\[0.25em]
        P_m(q) \, V_m(q), & q \in Q_m
    \end{cases}
    \label{eq:local-map-generation-reflectivity-fusion-fast} \\[0.5em]
\end{equation}
\begin{equation}
    P(q) = 
    \sum_{m \in S}
    \begin{cases}
        0, & q \notin Q_m \\[0.25em]
        P_m(q), & q \in Q_m
    \end{cases}
    \label{eq:local-map-generation-probability-sum-fast} \\[0.5em]
\end{equation}
\begin{equation}
    M(q) = \frac{V(q)}{P(q)}
    \label{eq:local-map-generation-final-map-fast} \\[0.5em]
\end{equation}
\begin{equation}
    M = \texttt{kNNFill}(M)
    \label{eq:local-map-generation-final-map-filled-inn-fast} \\[0.5em]
\end{equation}

\paragraph{\texttt{kNNFill(M)}} \mbox{}\\[0.5em] \noindent  
After generating the final probabilistic reflectivity map $M(q)$, a lightweight fill in step $\texttt{kNNFill}(M)$ is applied to ensure complete spatial coverage within the sonar footprint. This stage compensates for small unobserved gaps between adjacent swaths or at the edges of the footprint where the local probability values were too low to survive the final pruning stage.  
\\ \\
The fill in algorithm operates on the intermediate subset $Q_{m,\text{fill}}$, obtained after the second pruning stage but before probability thresholding, defined as:
\begin{equation}
    Q_{m,\text{fill}} = \left\{
        q \in Q \; \bigg| \;
        r_{\min} < \| \mathbf{p}_q - \mathbf{p}_{s/O}^{n}(m) \| < r_{\max},
        \;
        |\theta_{obs}(q)| < \frac{\alpha_h}{2}
    \right\}
    \label{eq:local-map-generation-fillin-domain}
\end{equation}
This subset includes all grid cells lying inside the sonars geometric beam limits, regardless of their final illumination probability. For each unobserved cell $q \in Q_{m,\text{fill}} \setminus Q_m$, the missing reflectivity value is estimated as the mean of its $K$ nearest valid neighbors within a predefined radius:
\begin{equation}
    M(q) = \frac{1}{K} \sum_{k=1}^{K} M(q_k)
    \label{eq:local-map-generation-post-processing-fillin-fast}
\end{equation}
where $q_k$ are neighboring cells already assigned valid reflectivity values from the probabilistic fusion stage.  
\\ \\
To efficiently identify all cells belonging to $Q_{m,\text{fill}}$, a flood fill method is employed instead of scanning the entire grid. The search begins from two seed cells known to be observed, one near the left and one near the right swath boundary, and expands recursively to neighboring cells until no new valid candidates are found. This approach minimizes computation while ensuring that only spatially relevant cells within the active sonar footprint are processed.  
\\ \\
The $\texttt{kNNFill}(M)$ procedure produces a visually and numerically continuous map by smoothly bridging small gaps and extending valid regions, resulting in a refined reflectivity field $M(q)$ suitable for downstream SLAM, navigation, or feature extraction tasks.

\paragraph{Optimized Algorithm} \mbox{}\\[0.5em] \noindent
At a high level, the complete accelerated probabilistic mapping algorithm proceeds as follows.  
\\ \\
Inputs and outputs are summarized first, followed by the main processing steps referencing the corresponding equations for clarity. 
\\ \noindent
\textbf{Input:}
\begin{itemize}
    \item Processed swath sonar samples $\mathcal{S} = \{\mathbf{p}_{g/O}^{n}(i), \hat{R}(i)\}_{i=1}^{N_s}$ in global coordinates.
    \item Vehicle position $\mathbf{p}_{b/O}^{n}$ and attitude $\mathbf{q}$ from state estimator.
    \item Grid definition $Q = \{q(x,y)\}$ describing local map cells.
\end{itemize}
\noindent
\textbf{Output:}
\begin{itemize}
    \item Final probabilistic reflectivity map $M(q)$, optionally post processed with $\texttt{kNNFill}(M)$ for spatial completeness.
\end{itemize}
\noindent
\textbf{Algorithm Steps:}
\begin{enumerate}
    \item \textbf{Three-Stage Pruning:}  
    Sequentially restrict computation to geometrically and probabilistically valid cells for each ping $m$, range filtering, angular filtering, and probability thresholding (Equation \ref{eq:local-map-generation-multistage-pruning-fast}). Resulting subset: $Q_m$

    \item \textbf{Interpolation:}  
    For all $q \in Q_m$, interpolate the 1D sonar samples $\hat{R}_m(i)$ to estimate local cell intensity $V_m(q)$ using bilinear interpolation (Equation \ref{eq:local-map-generation-interpolation-fast})

    \item \textbf{Reflectivity Fusion:}  
    Accumulate per ping reflectivity contributions weighted by illumination probability to form $V(q)$ and $P(q)$ (Equations \ref{eq:local-map-generation-reflectivity-fusion-fast} \ref{eq:local-map-generation-probability-sum-fast})

    \item \textbf{Normalization:}  
    Normalize fused reflectivity by cumulative probability to obtain the map estimate $M(q)$ (Equation \ref{eq:local-map-generation-final-map-fast})

    \item \textbf{Fill-In Post-Processing:}  
    Apply the lightweight $\texttt{kNNFill}(M)$ step (Equation \ref{eq:local-map-generation-final-map-filled-inn-fast})  
    using the subset $Q_{m,\text{fill}}$ (Equation \ref{eq:local-map-generation-fillin-domain}) to fill gaps via $K$ nearest neighbor averaging (Equation \ref{eq:local-map-generation-post-processing-fillin-fast})
\end{enumerate}
\noindent
This optimized implementation preserves the physical consistency of the full probabilistic model while reducing complexity from $\mathcal{O}(N_s N_g)$ to approximately $\mathcal{O}(S B + R C)$, enabling real-time mapping on embedded hardware.



\subsubsection{Sliding Map Window and Overlap}
The local probabilistic map represents a continuously updated portion of the seafloor constructed from the most recent $N$ processed sonar swaths. Initially, $N$ swaths are accumulated and converted into Cartesian coordinates to form the first local map.  
\\ \\
During operation, the map updates in a sliding window fashion: approximately one third of the previous data is retained for overlap, while the remaining two thirds are replaced by newly processed swaths. This overlap maintains spatial and temporal continuity between consecutive local maps, providing consistent feature visibility for SLAM data association and reducing discontinuities at map boundaries.  
\\ \\
The choice of $N$ determines the trade off between temporal resolution and mapping stability. A larger $N$ yields smoother, more stable maps but introduces higher latency and lower update rate, while a smaller $N$ increases temporal responsiveness at the cost of reduced spatial context and fewer observable landmarks. In practice, $N$ is tuned empirically based on the vehicles speed, ping rate, and sonar range, typically corresponding to a window of several hundred to a few thousand swaths.  
\\ \\
For real-time execution, the sliding map buffer is implemented as a \textit{``ring buffer''}, where new swaths overwrite the oldest entries once the capacity $N$ is reached. This structure maintains a constant memory footprint and allows efficient data replacement in $\mathcal{O}(1)$ time, ensuring deterministic performance even during long-duration missions.  
\\ \\
The sliding window mechanism ensures that the local probabilistic map always reflects the most recent portion of the seafloor while maintaining a controlled overlap with previous maps. This enables smooth temporal transitions, stable landmark tracking, and spatial continuity required for reliable SLAM operation.  
\\ \\
\begin{figure}[H]
    \centering
    \includegraphics[width=0.9\linewidth]{Pictures/Local_Map_Generation/Probabilistic_Mapping/Sliding_Map_Window.png}
    \caption{Sliding window mechanism for local map updates. Each frame represents a local probabilistic map composed of $N$ swaths. As new data arrives, the window advances while retaining one third of the previous samples for overlap, ensuring continuity and real-time operation through a ring buffer structure.}
    \label{fig:local-map-generation-sliding-map-window}
\end{figure}


