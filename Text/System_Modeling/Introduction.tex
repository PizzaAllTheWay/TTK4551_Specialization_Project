\subsection{Introduction}
Reliable navigation and mapping for an autonomous vessel depend on accurate mathematical models that describe how the vessel moves and how its sensors perceive the surrounding environment. These models form the foundation for state estimation, control, and SLAM algorithms, ensuring that all physical quantities such as position, velocity, and orientation are consistently defined and propagated over time.  
\\ \\
The modeling framework is divided into two complementary parts, kinematics and dynamics. Kinematics focuses on the geometric description of motion without considering the forces that cause it. It defines how position, velocity, and orientation are represented and transformed between coordinate frames. Global reference frames such as the World Geodetic System 1984 (WGS84) and the Earth Centered Earth Fixed (ECEF) frame describe the Earth geometry and provide absolute positioning. Local frames such as the North East Down (NED) and Body frames define the vessel motion relative to its own orientation and surroundings. The transformation chain between these frames ensures that all measured and estimated quantities are consistently expressed and can be converted between global and local coordinates.
\\ \\
Dynamics extend the kinematic framework by describing how forces and moments act on the vessel to generate motion. This relationship forms the basis of the motion models used in navigation and estimation. Two main formulations are considered. The Marine Craft Model by Thor Inge Fossen \cite{fossen_marine_craft_model} provides a comprehensive six degree of freedom (6 DOF) description that captures hydrodynamic effects, external disturbances, and control inputs, making it ideal for simulation and model based control. The Inertial Navigation System (INS) model, derived from the work of Edmund Brekke in Fundamentals of Sensor Fusion \cite{sensor_fusion_book}, offers a simplified sensor driven formulation that relies directly on IMU and GNSS data. This approach is better suited for real-time operation and SLAM applications, where computational efficiency and robustness to environmental uncertainty are prioritized over detailed hydrodynamic accuracy.  
\\ \\
The complete mathematical framework connects these kinematic and dynamic representations into a unified structure. It defines how vessel states are represented, transformed, and propagated in time using rotation formulations such as Euler angles, quaternions, and Lie group representations. It establishes consistent conventions between global and local reference frames and expresses the relationships between linear and angular motion necessary for deriving velocities, accelerations, and time derivatives of position and orientation.  
\\ \\
To achieve stable and accurate temporal evolution of the state, numerical solvers are used to integrate the underlying differential equations. Classical integration schemes such as Newton-Euler and Runge-Kutta methods are employed to propagate the vessel dynamics in discrete time while minimizing numerical drift and maintaining physical consistency. The choice of solver has a significant impact on model accuracy, especially in real-time applications where integration stability and computational efficiency determine overall system performance.  
\\ \\
Together, these formulations provide a rigorous mathematical foundation for navigation and mapping in autonomous surface vessels. They enable consistent state representation, accurate propagation, and reliable fusion of multi-sensor information, which are all essential for achieving robust autonomy and for the integration of advanced perception systems such as SSS SLAM.
