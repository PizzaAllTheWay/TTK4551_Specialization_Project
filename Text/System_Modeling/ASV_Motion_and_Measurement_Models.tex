\subsection{ASV Motion and Measurement Models}
\subsubsection{Overview}
This chapter presents the mathematical framework used to model the motion dynamics and sensor measurements of the ASV. Two main modeling approaches are discussed, the nonlinear 6 DOF marine craft model developed by Fossen \cite{fossen_marine_craft_model}, and the Inertial Navigation System (INS) based kinematic model described in Edmund Brekke book on Sensor Fusion \cite{sensor_fusion_book}. Both formulations provide complementary perspectives on vehicle motion, Fossen's model emphasizes a physically consistent hydrodynamic representation, while the INS model focuses on inertial propagation and sensor fusion suitable for real-time navigation.  
\\ \\
The chapter begins by introducing Fossen's marine craft model, which captures the rigid body and hydrodynamic behavior of marine vehicles through a compact matrix vector formulation that includes added mass, Coriolis, damping, and restoring effects. The INS based kinematic model is then presented as the practical implementation used in this work, describing the evolution of position, velocity, and attitude from inertial data while modeling sensor biases and stochastic errors. Aiding measurements from a dual antenna GNSS system are incorporated to correct drift and estimate heading. Although Fossen's model provides higher physical fidelity, it requires extensive parameter identification, whereas the INS model achieves sufficient accuracy for SLAM and local navigation when aided by GNSS. Thus, the INS based approach is chosen as the optimal balance between simplicity, computational efficiency, and practical performance.



\subsubsection{Kinetic Motion Model}
The kinetic motion model employed in this work is based on Fossens nonlinear 6 DOF marine craft formulation, which provides a unified mathematical framework for describing the dynamic behavior of marine vehicles \cite{fossen_marine_craft_model}. This model captures both rigid body and hydrodynamic effects using a compact matrix vector representation, making it suitable for analysis, control, and simulation of ships, underwater vehicles, and surface vessels. By bridging classical rigid body mechanics with hydrodynamic theory, the formulation offers a complete description of marine craft dynamics while preserving key physical properties such as symmetry, passivity, and energy consistency.
\\ \\
The general 6 DOF equations of motion can be written as
$$
    \mathbf{M}\dot{\boldsymbol{\nu}} + \mathbf{C}(\boldsymbol{\nu})\boldsymbol{\nu} + \mathbf{D}(\boldsymbol{\nu})\boldsymbol{\nu} + \mathbf{g}(\boldsymbol{\eta}) = \boldsymbol{\tau}
$$
where $\boldsymbol{\eta} = [x, y, z, \phi, \theta, \psi]^\top$ represents position and orientation, and $\boldsymbol{\nu} = [u, v, w, p, q, r]^\top$ denotes linear and angular velocities in the body frame. The vector $\boldsymbol{\tau}$ contains external forces and moments acting on the vessel from propulsion, wind, waves, and current. This formulation expresses the total balance between inertia, Coriolis, damping, and restoring forces on the left hand side, and all external excitations on the right hand side, forming the foundation for the dynamics of any marine craft.  
\\ \\
The inertia matrix $\mathbf{M}$ consists of both rigid body and added mass contributions,
$$
    \mathbf{M} = \mathbf{M}_{RB} + \mathbf{M}_A
$$
where $\mathbf{M}_{RB}$ represents the rigid body mass and inertia, while $\mathbf{M}_A$ captures the hydrodynamic added mass caused by the acceleration of surrounding water as the hull moves. The added mass terms are particularly important for underwater and high speed vehicles, as they significantly influence acceleration response and stability. This total inertia matrix determines how the craft resists changes in motion, acting as a coupling term between linear and angular accelerations.  
\\ \\
The Coriolis and centripetal effects are similarly expressed as
$$
    \mathbf{C}(\boldsymbol{\nu}) = \mathbf{C}_{RB}(\boldsymbol{\nu}) + \mathbf{C}_A(\boldsymbol{\nu})
$$
with $\mathbf{C}_{RB}$ and $\mathbf{C}_A$ representing the rigid body and added mass contributions, respectively. These matrices account for dynamic coupling between translational and rotational motion, such as how a turn induces sway or roll. The Coriolis matrix is skew-symmetric, ensuring that it does not contribute to net energy gain or loss, only redistributing kinetic energy among the motion axes.  
\\ \\
The matrix $\mathbf{D}(\boldsymbol{\nu})$ models hydrodynamic damping from viscous drag, wave radiation, and flow separation effects. It typically includes linear damping terms valid for small velocities and nonlinear or quadratic terms that dominate at higher speeds. The damping effects are always dissipative, converting kinetic energy into heat or wave energy, thus stabilizing the system over time.  
\\ \\
The restoring forces and moments $\mathbf{g}(\boldsymbol{\eta})$ describe gravitational and buoyancy effects, depending on the vehicles geometry and displacement. These terms act to return the vessel to equilibrium when displaced, defining roll, pitch, and heave stability through hydrostatic stiffness. For surface vessels, the restoring forces are strongly dependent on metacentric heights and the waterplane area, while for submerged vehicles, they depend mainly on the distance between the centers of gravity and buoyancy.  
\\ \\
The kinematic relationships between the body and NED frames are given by
$$
    \dot{\boldsymbol{\eta}} = \mathbf{J}(\boldsymbol{\eta})\boldsymbol{\nu}
$$
where $\mathbf{J}(\boldsymbol{\eta})$ is a block diagonal transformation matrix containing the rotation matrix $R_b^n$ for translational motion and $T(\phi, \theta)$ for angular motion. This mapping relates the velocity expressed in the body frame to the rate of change of position and attitude in the navigation frame, and can be implemented using either Euler angles or quaternions to avoid singularities.  
\\ \\
Environmental effects such as ocean currents, wind, and wave disturbances can be included by introducing the relative velocity $\boldsymbol{\nu}_r = \boldsymbol{\nu} - \boldsymbol{\nu}_c$, where $\boldsymbol{\nu}_c$ is the current velocity expressed in the body frame. This modifies the hydrodynamic forces and damping terms to account for the vehicles motion relative to the surrounding water, which is essential for accurate modeling of drift and current induced forces.  
\\ \\
For surface vessels, the equations can be simplified to the planar 3 DOF surge, sway, and yaw model:
$$
    \mathbf{M}\dot{\boldsymbol{\nu}} + \mathbf{C}(\boldsymbol{\nu})\boldsymbol{\nu} + \mathbf{D}(\boldsymbol{\nu})\boldsymbol{\nu} = \boldsymbol{\tau}
$$
where $\boldsymbol{\nu} = [u, v, r]^\top$. This reduced model captures the dominant motion in the horizontal plane and is widely used for uncrewed surface vehicles and control applications where roll, pitch, and heave motions are negligible.
\\ \\
The full nonlinear state space representation of Fossen's marine craft model can be written as
$$
    \dot{\mathbf{x}} = f(\mathbf{x}, \mathbf{u}) =
    \begin{bmatrix}
        \dot{\boldsymbol{\eta}} \\
        \dot{\boldsymbol{\nu}}
    \end{bmatrix}
    =
    \begin{bmatrix}
        \mathbf{J}(\boldsymbol{\eta})\boldsymbol{\nu} \\
        \mathbf{M}^{-1}\big(\boldsymbol{\tau} - \mathbf{C}(\boldsymbol{\nu})\boldsymbol{\nu} - \mathbf{D}(\boldsymbol{\nu})\boldsymbol{\nu} - \mathbf{g}(\boldsymbol{\eta})\big)
    \end{bmatrix}
$$
where
$$
    \mathbf{x} =
    \begin{bmatrix}
        \boldsymbol{\eta} & \boldsymbol{\nu}
    \end{bmatrix}^\top
    =
    [\,x,\, y,\, z,\, \phi,\, \theta,\, \psi,\, u,\, v,\, w,\, p,\, q,\, r\,]^\top,
    \qquad
    \mathbf{u} = \boldsymbol{\tau}
$$
Here, the first part of the function $f(\mathbf{x}, \mathbf{u})$ describes the kinematic mapping from body frame velocities to the time derivative of position and attitude through the transformation matrix $\mathbf{J}(\boldsymbol{\eta})$, while the second part represents the vehicle kinetics defined by Newton-Euler dynamics. The term $\mathbf{M}^{-1}$ acts as an inverse inertia operator, mapping the net generalized forces and moments (after accounting for Coriolis, damping, and restoring effects) to the acceleration in the body frame.  
\\ \\
This formulation provides a complete and compact representation of the vessels motion, directly linking control inputs and environmental disturbances to the time evolution of the vehicles state.
\\ \\
Fossen's formulation exhibits several key properties, the mass matrix $\mathbf{M}$ is symmetric and positive definite, the Coriolis matrix $\mathbf{C}(\boldsymbol{\nu})$ is skew-symmetric, and the damping matrix $\mathbf{D}(\boldsymbol{\nu})$ is positive definite. These properties guarantee passivity and energy conservation, ensuring that the system dissipates energy over time and remains physically consistent. This structure also provides a strong foundation for control system design and stability analysis, as it aligns with fundamental principles of energy based modeling and Lyapunov stability.



\subsubsection{Kinematic Motion Model}
The kinematic motion model adopted in this work is based on the Inertial Navigation System (INS) formulation, which describes the time evolution of a vehicles position, velocity, and attitude as functions of inertial sensor measurements \cite{sensor_fusion_book}. Unlike hydrodynamic models that depend on physical parameters such as mass, damping, or buoyancy, the INS model is purely kinematic and relies solely on accelerometer and gyroscope data. This independence from the vehicles geometry, operating environment, and fluid interaction makes it well suited for generic navigation and SLAM applications, where a simplified yet robust representation of motion is sufficient for accurate state propagation between aiding measurements.
\\ \\
IMUs typically represent attitude using quaternions rather than Euler angles. This avoids singularities and ensures smooth rotational updates even under large orientation changes, making quaternion based propagation the standard approach for INS implementations.  
\\ \\
The continuous time state vector is defined as
$$
    \mathbf{x} =
    \begin{bmatrix}
        \mathbf{p}_{b/O}^{n} & \mathbf{v}_{b/O}^{n} & \mathbf{q} & \mathbf{b}_a & \mathbf{b}_g
    \end{bmatrix}^\top
$$
where $\mathbf{p}_{b/O}^{n} = [x, y, z]^\top$ is the position of the body origin relative to the navigation frame, $\mathbf{v}_{b/O}^{n} = [v_x, v_y, v_z]^\top$ is the velocity in the same frame, $\mathbf{q}$ is the quaternion representing the rotation from body to navigation frame, and $\mathbf{b}_a$ and $\mathbf{b}_g$ are the accelerometer and gyroscope biases, respectively.  
\\ \\
The position and velocity states describe translational motion in the navigation frame, while the quaternion captures attitude evolution through rotational kinematics. The bias states account for the slowly varying sensor offsets that develop over time due to temperature changes, component aging, and other environmental factors. Together, these quantities form a complete minimal representation of the navigation state used in inertial systems, forming the basis for the continuous time kinematic model that governs their time evolution.  
\\ \\
The inertial navigation model therefore describes the kinematic evolution of the vehicles position, velocity, and attitude as driven by inertial sensor measurements. The foundation of this formulation is the deterministic, noise free relationship between translational and rotational motion, which forms the core of all INS propagation models.  
\\ \\
In the ideal case without any sensor errors, the motion of the sensor can be expressed as
$$
\begin{aligned}
    \dot{\mathbf{p}}_{s/O}^{n} &= \mathbf{v}_{s/O}^{n} \\
    \dot{\mathbf{v}}_{s/O}^{n} &= R_b^n(\mathbf{q})\,(R_s^b\,\mathbf{a}_m) + \mathbf{g}^n \\
    \dot{\mathbf{q}} &= \tfrac{1}{2}\,\mathbf{q} \otimes (R_s^b\,\boldsymbol{\omega}_m)
\end{aligned}
$$
where $\mathbf{p}_{s/O}^{n}$ and $\mathbf{v}_{s/O}^{n}$ are the position and velocity of the sensor with respect to an inertial reference frame like NED, $R_b^n(\mathbf{q})$ is the rotation matrix transforming vectors from the body frame to the navigation frame, and $R_s^b$ is the rotation matrix from the sensor to body frame. The measured quantities $\mathbf{a}_m$ and $\boldsymbol{\omega}_m$ are the specific force and angular rate readings provided by the IMU, expressed in the sensors own coordinate frame. These measurements are therefore related to the true body frame motion through the sensors fixed mounting orientation, such that $\mathbf{a}_m = \mathbf{a}_{s/b}^{s}$ and $\boldsymbol{\omega}_m = \boldsymbol{\omega}_{s/b}^{s}$.  
\\ \\
In practice, the inertial sensors are rarely placed exactly at the vehicles center of gravity (COG). When the sensor is offset by a position vector $\mathbf{r}_{s/b}^{b}$ from the body frame origin usually defined at the COG, the accelerations measured by the IMU no longer correspond directly to the translational accelerations of the vehicle. Instead, the rotational motion of the body introduces additional terms due to centripetal and tangential accelerations. The total acceleration of the sensor can then be expressed as
$$
    \mathbf{a}_{s/O}^{b} = \mathbf{a}_{b/O}^{b} + \boldsymbol{\alpha}_{b/O}^{b} \times \mathbf{r}_{s/b}^{b} + \boldsymbol{\omega}_{b/O}^{b} \times (\boldsymbol{\omega}_{b/O}^{b} \times \mathbf{p}_{s/b}^{b})
$$
where $\mathbf{a}_{b/O}^{b}$ is the linear acceleration of the body origin, $\boldsymbol{\omega}_{b/O}^{b}$ is the angular velocity of the body, and $\boldsymbol{\alpha}_{b/O}^{b}$ its angular acceleration. The second term represents the tangential acceleration component caused by rotational acceleration, while the third term corresponds to the centripetal acceleration arising from constant rotation. These effects are collectively known as the \textit{``lever arm effect''}, and they cause the IMU to measure accelerations that differ from those experienced at the COG.  
\\ \\
Accurately compensating for the lever arm effect requires knowledge of the exact mounting offset $\mathbf{p}_{s/b}^{b}$ and precise estimates of both angular velocity and angular acceleration. The latter is particularly difficult to obtain in real time, as most MEMS-based IMU (like microAmpere IMU) do not measure angular acceleration directly. Numerical differentiation of gyroscope signals amplifies measurement noise and introduces additional uncertainty, making real-time compensation computationally challenging and often unreliable.  
\\ \\
Therefore, in this master thesis work, it is assumed that the inertial sensors are mounted close to the vehicles center of gravity. For the microAmpere ASV, this assumption is valid since the IMU is physically located near the COG within the central electronics housing. As a result, the lever arm terms become negligible, and the simplified kinematic model given above provides an accurate and practical representation of the vehicle motion. This assumption greatly simplifies the mathematical model while retaining sufficient fidelity for navigation and SLAM applications.  
\\ \\
Under this assumption, the IMU can be considered approximately coincident with the body frame origin, effectively treating the sensor as rigidly aligned with the vehicles center of gravity. The motion model can then be expressed directly in terms of the body frame as
$$
\begin{aligned}
    \dot{\mathbf{p}}_{b/O}^{n} &= \mathbf{v}_{b/O}^{n} \\
    \dot{\mathbf{v}}_{b/O}^{n} &= R_b^n(\mathbf{q})\,\mathbf{a}_m + \mathbf{g}^n \\
    \dot{\mathbf{q}} &= \tfrac{1}{2}\,\mathbf{q} \otimes \boldsymbol{\omega}_m
\end{aligned}
$$
which defines the nominal, noise free kinematic model used in this work. It captures the translational and rotational motion of the ASV based solely on the IMUs accelerometer and gyroscope measurements, with attitude expressed through quaternion integration for smooth and singularity free orientation propagation.
\\ \\
In practice, the accelerometer and gyroscope measurements are not perfect and are affected by sensor biases and measurement noise. These imperfections cause the estimated position and attitude to drift over time, as small integration errors accumulate during motion. The measured sensor outputs can therefore be modeled as
$$
\begin{aligned}
    \mathbf{a}_m &= R^{T}(q_t)\,(\mathbf{a_t} - \mathbf{g}) + \mathbf{a}_{bt} + \mathbf{a}_{n} \\
    \boldsymbol{\omega}_m &= \boldsymbol{\omega}_t + \mathbf{\omega}_{bt} + \mathbf{\omega}_{n}
\end{aligned}
$$
where $\mathbf{a}_m$ and $\boldsymbol{\omega}_m$ are the measured accelerations and angular rates, $\mathbf{a}_{bt}$ and $\mathbf{\omega}_{bt}$ denote slowly varying sensor biases, and $\mathbf{a}_{n}$ and $\mathbf{\omega}_{n}$ are zero mean Gaussian noise processes representing measurement uncertainty. The reason for choosing zero mean Gaussian noise processes representation is just to make calculations easier to manage, and usually inertial sensor noise can be approximated to a pure gaussian.  
\\ \\
While the previous kinematic model assumed ideal, noise free measurements, real inertial sensors inherently produce signals corrupted by both bias and random noise. These effects introduce long term drift in velocity, position, and orientation estimates if uncorrected. To account for this, the noise and bias terms are explicitly included in the dynamic equations.  
\\ \\
By substituting the noisy sensor models into the noise free kinematics, we obtain the full continuous time INS motion model that represents the true system behavior under realistic sensor conditions:
$$
\begin{aligned}
    \dot{\mathbf{p}}_{b/O}^{n} &= \mathbf{v}_{b/O}^{n} \\
    \dot{\mathbf{v}}_{b/O}^{n} &= R_b^n(\mathbf{q})\,(\mathbf{a}_m - \mathbf{a}_{bt} - \mathbf{a}_n) + \mathbf{g}^n \\
    \dot{\mathbf{q}} &= \tfrac{1}{2}\,\mathbf{q} \otimes (\boldsymbol{\omega}_m - \mathbf{\omega}_{bt} - \mathbf{\omega}_n)
\end{aligned}
$$
where $R_b^n(\mathbf{q})$ transforms body frame accelerations into the navigation frame according to the current orientation quaternion.  
\\ \\
This formulation captures the complete inertial motion propagation process, accounting for both deterministic motion and stochastic sensor effects. It forms the basis for all subsequent modeling of bias dynamics and sensor noise processes.
\\ \\
Inertial measurement systems such as accelerometers and gyroscopes inherently suffer from drift over time, as they measure relative motion rather than absolute quantities. Small integration errors in acceleration and angular rate accumulate, causing increasing uncertainty in position and attitude estimates. To account for this phenomenon, bias terms are introduced to represent slow, time varying offsets in the sensor readings. Accurate bias modeling is essential for achieving reliable and stable navigation performance, particularly for low cost MEMS-based sensors where bias instability is a dominant error source.  
\\ \\
Sensor biases are typically represented as stochastic processes that evolve gradually over time. Several modeling approaches exist, each offering a trade off between realism and complexity. The simplest approach is the constant bias model,
$$
    \dot{\mathbf{b}} = 0
$$
which assumes a fixed, time invariant offset in the sensor readings. Although straightforward, this model is generally insufficient for practical applications since real world sensors exhibit continuous drift due to thermal, mechanical, and electronic variations.  
\\ \\
A more flexible formulation is the Wiener process, or better known as random walk model,
$$
    \dot{\mathbf{b}} = \mathbf{b}_n
$$
where $\mathbf{b}_n$ is a zero mean white noise process. This allows the bias to evolve stochastically over time, capturing unmodeled variations in sensor output. However, this model tends to be overly conservative, as the bias uncertainty grows without bound, making it unsuitable for long duration navigation where bounded behavior is required.  
\\ \\
A more realistic representation for modern inertial sensors is the first-order Gauss-Markov process,
$$
    \dot{\mathbf{b}} = -\tfrac{1}{\tau}\mathbf{b} + \mathbf{b}_n
$$
where $\tau$ is the correlation time constant that defines how quickly the bias decays toward zero. This model captures both short term fluctuations and long term drift, providing a balanced compromise between physical accuracy and numerical stability. It ensures that bias uncertainty remains bounded while reflecting the natural stochastic variation of real IMU behavior.  
\\ \\
Accordingly, the accelerometer and gyroscope biases are modeled as first order Gauss-Markov processes:
$$
\begin{aligned}
    \dot{\mathbf{a}}_{bt} &= -p_{\mathbf{a}b}\,I\,\mathbf{a}_{bt} + \mathbf{a}_{w} \\
    \dot{\mathbf{\omega}}_{bt} &= -p_{\mathbf{\omega}b}\,I\,\mathbf{\omega}_{bt} + \mathbf{\omega}_{w}
\end{aligned}
$$
where $p_{\mathbf{a}b} = 1/\tau_a$ and $p_{\mathbf{\omega}b} = 1/\tau_g$ denote the inverse correlation times for the accelerometer and gyroscope bias models, respectively. The terms $\mathbf{a}_{w}$ and $\mathbf{\omega}_{w}$ represent zero mean white noise processes driving the bias evolution.  
\\ \\
This bias modeling framework provides a physically consistent and numerically stable representation of inertial sensor drift. By incorporating these bias dynamics, the INS model can accurately describe both the deterministic vehicle motion and the stochastic effects introduced by the inertial sensors.  
\\ \\
With the inclusion of bias dynamics, the complete continuous time true state kinematic model of the system is expressed as
$$
\begin{aligned}
    \dot{\mathbf{p}}_{b/O}^{n} &= \mathbf{v}_{b/O}^{n} \\
    \dot{\mathbf{v}}_{b/O}^{n} &= R_b^n(\mathbf{q})\,(\mathbf{a}_m - \mathbf{a}_{bt} - \mathbf{a}_n) + \mathbf{g}^n \\
    \dot{\mathbf{q}} &= \tfrac{1}{2}\,\mathbf{q} \otimes (\boldsymbol{\omega}_m - \mathbf{\omega}_{bt} - \mathbf{\omega}_n) \\
    \dot{\mathbf{a}}_{bt} &= -p_{\mathbf{a}b}\,I\,\mathbf{a}_{bt} + \mathbf{a}_{w} \\
    \dot{\mathbf{\omega}}_{bt} &= -p_{\mathbf{\omega}b}\,I\,\mathbf{\omega}_{bt} + \mathbf{\omega}_{w}
\end{aligned}
$$
This formulation constitutes the complete nonlinear inertial motion model, capturing both the deterministic kinematics of the vehicle and the stochastic processes that characterize real sensor behavior.
\\ \\
The complete INS model previously presented captures the full physical and stochastic behavior of the vehicle, incorporating both deterministic motion and random processes such as measurement noise and bias drift. While this representation accurately reflects real sensor behavior, it is often desirable to isolate the deterministic part of the dynamics to describe the nominal system behavior. This simplified formulation, known as the nominal state kinematic model, represents how the system would evolve in the absence of stochastic disturbances. It provides a clear and noise-free description of the vehicles motion based purely on inertial measurements, forming the foundation for state propagation in navigation and estimation frameworks.  
\\ \\
The nominal model assumes that all noise components are zero, leaving only the essential deterministic relationships between position, velocity, attitude, and sensor biases. In this formulation, the accelerometer and gyroscope measurements are treated as control inputs driving the systems motion. This perspective simplifies the dynamics while preserving the physical structure of the original model, allowing efficient and accurate propagation of the navigation states between external aiding updates.  
\\ \\
By removing the stochastic noise terms from the true state equations, the nominal state kinematic model becomes
\begin{equation}
    \dot{\mathbf{x}} = f(\mathbf{x}, \mathbf{u}) =
    \begin{bmatrix}
        \dot{\mathbf{p}}_{b/O}^{n} \\
        \dot{\mathbf{v}}_{b/O}^{n} \\
        \dot{\mathbf{q}} \\
        \dot{\mathbf{a}}_b \\
        \dot{\mathbf{\omega}}_b
    \end{bmatrix}
    =
    \begin{bmatrix}
        \mathbf{v}_{b/O}^{n} \\
        R_b^n(\mathbf{q})\,(\mathbf{a}_m - \mathbf{a}_{b}) + \mathbf{g}^n \\
        \tfrac{1}{2}\,\mathbf{q} \otimes (\boldsymbol{\omega}_m - \mathbf{\omega}_{b}) \\
        -p_{\mathbf{a}b}\,I\,\mathbf{a}_b \\
        -p_{\mathbf{\omega}b}\,I\,\mathbf{\omega}_b
    \end{bmatrix}
    \label{eq:kinematics-motion-model}
\end{equation}
where
\begin{equation}
    \mathbf{x} =
    \begin{bmatrix}
        \mathbf{p}_{b/O}^{n} & \mathbf{v}_{b/O}^{n} & \mathbf{q} & \mathbf{a}_b & \mathbf{\omega}_b
    \end{bmatrix}^\top,
    \qquad
    \mathbf{u} =
    \begin{bmatrix}
        \mathbf{a}_m & \boldsymbol{\omega}_m
    \end{bmatrix}^\top
    \label{eq:kinematics-motion-model-states}
\end{equation}
Here, $\mathbf{p}_{b/O}^{n}$ and $\mathbf{v}_{b/O}^{n}$ represent the vehicles position and velocity expressed in the navigation frame, $\mathbf{q}$ denotes the attitude quaternion defining the rotation from the body to the navigation frame, and $\mathbf{a}_b$ and $\mathbf{\omega}_b$ represent the accelerometer and gyroscope biases, respectively. The parameters $p_{\mathbf{a}b} = 1/\tau_a$ and $p_{\mathbf{\omega}b} = 1/\tau_g$ correspond to the inverse correlation times governing the exponential decay rates of the bias dynamics.  
\\ \\
This nominal model represents the idealized, noise free motion of the vehicle as inferred from IMU measurements. It captures the deterministic propagation of position, velocity, attitude, and bias under perfect sensor conditions, serving as the baseline dynamic model for integrated navigation systems.  
\\ \\
By separating deterministic kinematics from stochastic sensor effects, the INS formulation provides a structured foundation for real-time state estimation. This nominal model is particularly suited for use within an Error State Kalman Filter (ESKF), where it defines the expected system evolution between measurement updates and enables correction using external aiding sources such as GNSS or magnetometers.  
\\ \\
Overall, the INS based motion model offers a compact and computationally efficient representation of the ASVs dynamics. While it omits detailed hydrodynamic effects, its balance of simplicity, generality, and compatibility with sensor fusion frameworks makes it a robust and practical choice for autonomous surface vehicle navigation and SLAM applications.



\subsubsection{Aiding Measurement Model}
The aiding measurements are obtained from a dual antenna GNSS system that provides absolute position and heading information in the NED frame. The two antennas are rigidly mounted on the bow, one on the port side and one on the starboard side, at known offsets from the vessels COG. This setup enables direct estimation of both the vessels global position and its yaw orientation, offering an absolute reference that continuously corrects the drift accumulated in the INS.  
\\ \\
Accurate aiding of position and attitude is crucial for maintaining reliable long term navigation performance. While the INS can propagate the motion of the vessel accurately over short time spans, it inevitably accumulates errors due to sensor noise and bias drift. The GNSS updates counteract this drift by providing absolute position corrections, while the dual antenna baseline yields a precise heading measurement that stabilizes the yaw estimate, one of the most critical states for surface vessel guidance and control. Reliable yaw information directly improves trajectory tracking, waypoint following, and steering stability.  
\\ \\
For autonomous surface vessels like the microAmpere, roll and pitch angles can be neglected due to the platforms inherently stable catamaran style hull. The dual pontoons provide strong passive stability and a nearly constant deck attitude even under moderate disturbances. Consequently, the vessel motion is effectively modeled in two dimensions, considering surge, sway, and yaw, while the vertical position $z$ is retained from GNSS for completeness. This simplification reduces system complexity without sacrificing accuracy and aligns well with the physical behavior and operational characteristics of small, inherently stable surface craft.
\\ \\
Each GNSS receiver provides a position measurement expressed in the navigation frame as
$$
    \mathbf{z}_{p,i} = \mathbf{p}_{\text{GNSS}i/O}^{n}
    = \mathbf{p}_{b/O}^{n} + R_b^n(\mathbf{q})\,\mathbf{p}_{\text{GNSS}i/b}^{b} + \mathbf{n}_{p,i}, 
    \qquad i \in \{1,2\}
$$
where $\mathbf{p}_{b/O}^{n}$ denotes the body origin (COG) position in the navigation frame, $\mathbf{p}_{\text{GNSS}i/b}^{b}$ are the known lever arm offsets of each antenna expressed in the body frame, and $\mathbf{n}_{p,i}$ represents zero mean Gaussian measurement noise. Since both antennas are positioned forward of the COG, this transformation is necessary to align the raw GNSS positions with the INS state representation.  
\\ \\
From these two absolute position measurements, the relative vector between the antennas can be expressed in the navigation frame as
$$
    \mathbf{r}_{1/2}^{n} = \mathbf{r}_{GNSS1/GNSS2}^{n} = \mathbf{p}_{\text{GNSS2}/O}^{n} - \mathbf{p}_{\text{GNSS1}/O}^{n}
$$
This baseline vector defines the spatial relationship between the antennas and allows extraction of the vessels yaw angle $\psi$ through its horizontal components as
$$
    \psi = \text{atan2}(r_{1/2,y}^{n},\, r_{1/2,x}^{n})
$$
Here, $\psi$ represents the heading of the line connecting the port and starboard antennas relative to true north. The accuracy of the heading estimate depends on the baseline length, where a longer lateral separation provides greater sensitivity and higher precision. The dual antenna setup therefore yields a robust and drift free heading measurement that complements the attitude propagation performed by the INS.  
\\ \\
The two GNSS position measurements and the derived yaw angle are then combined into a single aiding measurement vector that provides both global position and orientation information. The complete GNSS measurement is expressed as
$$
    \mathbf{z}_{\text{GNSS}} =
    \begin{bmatrix}
        \mathbf{p}_{b/O}^{n} \\
        \psi
    \end{bmatrix}
    = h(\mathbf{x}) + \mathbf{n}_{\text{GNSS}}
$$
where the nonlinear measurement function $h(\mathbf{x})$ relates the navigation state
$$
    \mathbf{x} =
    \begin{bmatrix}
        \mathbf{p}_{b/O}^{n} & \mathbf{v}_{b/O}^{n} & \mathbf{q} & \mathbf{b}_a & \mathbf{b}_g
    \end{bmatrix}^\top
$$
to the observed quantities as
\begin{equation}
    h(\mathbf{x}) =
    \begin{bmatrix}
        \mathbf{p}_{b/O}^{n} + R_b^n(\mathbf{q})\,\mathbf{p}_{\text{GNSS1}/b}^{b} \\
        \text{atan2}\big((R_b^n(\mathbf{q})\,\mathbf{r}_{1/2}^{b})_y,\,(R_b^n(\mathbf{q})\,\mathbf{r}_{1/2}^{b})_x\big)
    \end{bmatrix}
    \label{eq:aiding-measurement-model}
\end{equation}
In this expression, the first component corresponds to the transformed GNSS position measurement, while the second term computes the expected yaw angle $\psi$ from the vehicle attitude quaternion $\mathbf{q}$ and the known baseline vector $\mathbf{r}_{1/2}^{b}$ between antennas in the body frame. The term $\mathbf{n}_{\text{GNSS}}$ represents zero mean Gaussian noise, combining both position and heading measurement uncertainties, with covariance $\mathbf{R}_{\text{GNSS}} = \text{diag}(\mathbf{R}_{p}, \sigma_\psi^2)$.
\\ \\
For small autonomous surface vessels such as the MicroAmpere, the platform demonstrates inherently stable roll and pitch dynamics due to its catamaran style pontoon design. This hydrodynamic stability allows the assumption that roll and pitch angles remain close to zero during nominal operation, effectively reducing the attitude representation to a single dominant rotation about the vertical axis, corresponding to the yaw angle. With this simplification, the GNSS aiding measurements primarily correct horizontal position and heading, while small vertical displacements and tilting motions are neglected, as they have minimal impact on overall navigation accuracy.  
\\ \\
The dual antenna GNSS configuration therefore provides a compact and robust aiding source that supplies both absolute position and heading information in the navigation frame. By constraining the inertial navigation solution with globally referenced measurements, it compensates for the natural drift of the INS and ensures long-term stability, accuracy, and global consistency in the estimated vehicle pose. This makes the aiding model particularly suitable for marine environments where attitude variations are small and precise horizontal localization is of primary importance.



\subsubsection{Model Comparison and Selection}
The Fossen marine craft kinetic model and the INS kinematic model represent two complementary approaches to describing marine vehicle motion, positioned at opposite ends of the modeling spectrum between kinetic and kinematic formulations. The kinetic, or dynamic, model proposed by Fossen provides a complete physical representation of the vessels motion by explicitly accounting for rigid body dynamics, hydrodynamic effects, added mass, damping, and restoring forces. This yields a high fidelity model suitable for control design, simulation, and performance analysis under varying sea conditions. However, this level of detail comes with considerable complexity, as the model depends on numerous hydrodynamic and inertial parameters that must be carefully identified or experimentally measured.  
\\ \\
Conversations with experts in system identification, such as Shenglun Yi from the University of Padova, highlight that while partial parameter identification using recursive or adaptive methods is feasible, obtaining a complete and accurate hydrodynamic parameter set is extremely difficult in practice. Only a few parameters, such as surge damping or longitudinal added mass, can be reliably identified through simplified tests, while the majority require extensive experimentation and highly controlled conditions. As a result, performing full hydrodynamic system identification for small autonomous vessels like the microAmpere would demand substantial effort with limited improvement in navigation performance, making it impractical for this application.  
\\ \\
On the other hand, the INS based kinematic model provides a simplified but effective alternative. It represents the vessels motion through inertial sensor data, accelerations and angular rates, without explicitly modeling the underlying dynamics or environmental interactions. This makes it computationally efficient and easily implementable, while still providing accurate short term state propagation suitable for SLAM, navigation, and sensor fusion tasks. The model inherently focuses on position, velocity, and attitude evolution, allowing seamless integration with aiding sensors such as GNSS or vision based systems.  
\\ \\
The INS formulation, however, also introduces certain assumptions. One key simplification adopted in this work is that the IMU is positioned close to the vessels COG, effectively neglecting the lever arm effects that arise from angular accelerations acting on sensors offset from the COG. For the microAmpere, this assumption is valid since the IMU is mounted sufficiently close to the COG, minimizing the induced linear acceleration errors and simplifying the overall model structure.  
\\ \\
In conclusion, the INS based kinematic model occupies a balanced position between dynamic completeness and modeling simplicity. It omits the complex hydrodynamic modeling of the kinetic approach but retains sufficient fidelity for accurate state estimation and SLAM applications. For small, stable surface vessels, this trade off provides the optimal compromise between accuracy, computational efficiency, and practical feasibility, making the INS based formulation the preferred choice for this thesis.


