\subsection{ASV Motion and Measurement Models}
\subsubsection{Overview}
This chapter presents the mathematical framework used to model the motion dynamics and sensor measurements of the ASV. Two main modeling approaches are discussed, the nonlinear 6 DOF marine craft model developed by Fossen \cite{fossen_marine_craft_model}, and the Inertial Navigation System (INS) based kinematic model described in Edmund Brekke book on Sensor Fusion \cite{sensor_fusion_book}. Both formulations provide complementary perspectives on vehicle motion, Fossen's model emphasizes a physically consistent hydrodynamic representation, while the INS model focuses on inertial propagation and sensor fusion suitable for real-time navigation.  
\\ \\
The chapter begins by introducing Fossen's marine craft model, which captures the rigid body and hydrodynamic behavior of marine vehicles through a compact matrix vector formulation that includes added mass, Coriolis, damping, and restoring effects. The INS based kinematic model is then presented as the practical implementation used in this work, describing the evolution of position, velocity, and attitude from inertial data while modeling sensor biases and stochastic errors. Aiding measurements from a dual antenna GNSS system are incorporated to correct drift and estimate heading. Although Fossen's model provides higher physical fidelity, it requires extensive parameter identification, whereas the INS model achieves sufficient accuracy for SLAM and local navigation when aided by GNSS. Thus, the INS based approach is chosen as the optimal balance between simplicity, computational efficiency, and practical performance.



\subsubsection{Fossen's Marine Craft Model as Motion Model}
Fossen's marine craft model provides a unified mathematical framework for describing the nonlinear motion dynamics of marine vehicles. The formulation captures both rigid body and hydrodynamic effects using a compact matrix vector representation, making it suitable for analysis, control, and simulation of ships, underwater vehicles, and surface vessels \cite{fossen_marine_craft_model}. It effectively bridges classical rigid body mechanics and hydrodynamic theory, allowing a complete description of marine vehicle motion in 6 DOF while preserving important system properties such as symmetry and passivity.  
\\ \\
The general 6 DOF equations of motion can be written as
$$
    \mathbf{M}\dot{\boldsymbol{\nu}} + \mathbf{C}(\boldsymbol{\nu})\boldsymbol{\nu} + \mathbf{D}(\boldsymbol{\nu})\boldsymbol{\nu} + \mathbf{g}(\boldsymbol{\eta}) = \boldsymbol{\tau}
$$
where $\boldsymbol{\eta} = [x, y, z, \phi, \theta, \psi]^\top$ represents position and orientation, and $\boldsymbol{\nu} = [u, v, w, p, q, r]^\top$ denotes linear and angular velocities in the body frame. The vector $\boldsymbol{\tau}$ contains external forces and moments acting on the vessel from propulsion, wind, waves, and current. This formulation expresses the total balance between inertia, Coriolis, damping, and restoring forces on the left hand side, and all external excitations on the right hand side, forming the foundation for the dynamics of any marine craft.  
\\ \\
The inertia matrix $\mathbf{M}$ consists of both rigid body and added mass contributions,
$$
    \mathbf{M} = \mathbf{M}_{RB} + \mathbf{M}_A
$$
where $\mathbf{M}_{RB}$ represents the rigid body mass and inertia, while $\mathbf{M}_A$ captures the hydrodynamic added mass caused by the acceleration of surrounding water as the hull moves. The added mass terms are particularly important for underwater and high speed vehicles, as they significantly influence acceleration response and stability. This total inertia matrix determines how the craft resists changes in motion, acting as a coupling term between linear and angular accelerations.  
\\ \\
The Coriolis and centripetal effects are similarly expressed as
$$
    \mathbf{C}(\boldsymbol{\nu}) = \mathbf{C}_{RB}(\boldsymbol{\nu}) + \mathbf{C}_A(\boldsymbol{\nu})
$$
with $\mathbf{C}_{RB}$ and $\mathbf{C}_A$ representing the rigid body and added mass contributions, respectively. These matrices account for dynamic coupling between translational and rotational motion, such as how a turn induces sway or roll. The Coriolis matrix is skew-symmetric, ensuring that it does not contribute to net energy gain or loss, only redistributing kinetic energy among the motion axes.  
\\ \\
The matrix $\mathbf{D}(\boldsymbol{\nu})$ models hydrodynamic damping from viscous drag, wave radiation, and flow separation effects. It typically includes linear damping terms valid for small velocities and nonlinear or quadratic terms that dominate at higher speeds. The damping effects are always dissipative, converting kinetic energy into heat or wave energy, thus stabilizing the system over time.  
\\ \\
The restoring forces and moments $\mathbf{g}(\boldsymbol{\eta})$ describe gravitational and buoyancy effects, depending on the vehicles geometry and displacement. These terms act to return the vessel to equilibrium when displaced, defining roll, pitch, and heave stability through hydrostatic stiffness. For surface vessels, the restoring forces are strongly dependent on metacentric heights and the waterplane area, while for submerged vehicles, they depend mainly on the distance between the centers of gravity and buoyancy.  
\\ \\
The kinematic relationships between the body and NED frames are given by
$$
    \dot{\boldsymbol{\eta}} = \mathbf{J}(\boldsymbol{\eta})\boldsymbol{\nu}
$$
where $\mathbf{J}(\boldsymbol{\eta})$ is a block diagonal transformation matrix containing the rotation matrix $R_b^n$ for translational motion and $T(\phi, \theta)$ for angular motion. This mapping relates the velocity expressed in the body frame to the rate of change of position and attitude in the navigation frame, and can be implemented using either Euler angles or quaternions to avoid singularities.  
\\ \\
Environmental effects such as ocean currents, wind, and wave disturbances can be included by introducing the relative velocity $\boldsymbol{\nu}_r = \boldsymbol{\nu} - \boldsymbol{\nu}_c$, where $\boldsymbol{\nu}_c$ is the current velocity expressed in the body frame. This modifies the hydrodynamic forces and damping terms to account for the vehicles motion relative to the surrounding water, which is essential for accurate modeling of drift and current induced forces.  
\\ \\
For surface vessels, the equations can be simplified to the planar 3 DOF surge, sway, and yaw model:
$$
    \mathbf{M}\dot{\boldsymbol{\nu}} + \mathbf{C}(\boldsymbol{\nu})\boldsymbol{\nu} + \mathbf{D}(\boldsymbol{\nu})\boldsymbol{\nu} = \boldsymbol{\tau}
$$
where $\boldsymbol{\nu} = [u, v, r]^\top$. This reduced model captures the dominant motion in the horizontal plane and is widely used for uncrewed surface vehicles and control applications where roll, pitch, and heave motions are negligible.
\\ \\
The full nonlinear state space representation of Fossen's marine craft model can be written as
$$
    \dot{\mathbf{x}} = f(\mathbf{x}, \mathbf{u}) =
    \begin{bmatrix}
        \dot{\boldsymbol{\eta}} \\
        \dot{\boldsymbol{\nu}}
    \end{bmatrix}
    =
    \begin{bmatrix}
        \mathbf{J}(\boldsymbol{\eta})\boldsymbol{\nu} \\
        \mathbf{M}^{-1}\big(\boldsymbol{\tau} - \mathbf{C}(\boldsymbol{\nu})\boldsymbol{\nu} - \mathbf{D}(\boldsymbol{\nu})\boldsymbol{\nu} - \mathbf{g}(\boldsymbol{\eta})\big)
    \end{bmatrix}
$$
where
$$
    \mathbf{x} =
    \begin{bmatrix}
        \boldsymbol{\eta} & \boldsymbol{\nu}
    \end{bmatrix}^\top
    =
    [\,x,\, y,\, z,\, \phi,\, \theta,\, \psi,\, u,\, v,\, w,\, p,\, q,\, r\,]^\top,
    \qquad
    \mathbf{u} = \boldsymbol{\tau}
$$
Here, the first part of the function $f(\mathbf{x}, \mathbf{u})$ describes the kinematic mapping from body frame velocities to the time derivative of position and attitude through the transformation matrix $\mathbf{J}(\boldsymbol{\eta})$, while the second part represents the vehicle kinetics defined by Newton-Euler dynamics. The term $\mathbf{M}^{-1}$ acts as an inverse inertia operator, mapping the net generalized forces and moments (after accounting for Coriolis, damping, and restoring effects) to the acceleration in the body frame.  
\\ \\
This formulation provides a complete and compact representation of the vessels motion, directly linking control inputs and environmental disturbances to the time evolution of the vehicles state.
\\ \\
Fossen's formulation exhibits several key properties, the mass matrix $\mathbf{M}$ is symmetric and positive definite, the Coriolis matrix $\mathbf{C}(\boldsymbol{\nu})$ is skew-symmetric, and the damping matrix $\mathbf{D}(\boldsymbol{\nu})$ is positive definite. These properties guarantee passivity and energy conservation, ensuring that the system dissipates energy over time and remains physically consistent. This structure also provides a strong foundation for control system design and stability analysis, as it aligns with fundamental principles of energy based modeling and Lyapunov stability.



\subsubsection{Inertial Navigation System (INS) Model as Motion Model}
The inertial navigation model describes the kinematic evolution of a vehicles position, velocity, and attitude as driven by inertial sensor measurements \cite{sensor_fusion_book}. Unlike hydrodynamic models that depend on physical parameters such as mass, damping, or buoyancy, the INS formulation is purely kinematic and relies only on accelerometer and gyroscope data. This makes it independent of the vehicles geometry, operating environment, or fluid interaction. As a result, the model is particularly suitable for generic navigation and SLAM applications where dynamic modeling of external forces is unnecessary.  
\\ \\
IMUs typically represent attitude using quaternions rather than Euler angles. This avoids singularities and ensures smooth rotational updates even under large orientation changes, making quaternion based propagation the standard approach for INS implementations.  
\\ \\
The continuous time state vector is defined as
$$
    \mathbf{x} =
    \begin{bmatrix}
        \mathbf{p}_{b/O}^{n} & \mathbf{v}_{b/O}^{n} & \mathbf{q} & \mathbf{b}_a & \mathbf{b}_g
    \end{bmatrix}^\top
$$
where $\mathbf{p}_{b/O}^{n} = [x, y, z]^\top$ is the position of the body origin relative to the navigation frame, $\mathbf{v}_{b/O}^{n} = [v_x, v_y, v_z]^\top$ is the velocity in the same frame, $\mathbf{q}$ is the quaternion representing the rotation from body to navigation frame, and $\mathbf{b}_a$ and $\mathbf{b}_g$ are the accelerometer and gyroscope biases, respectively.  
\\ \\
The position and velocity states describe translational motion in the navigation frame, while the quaternion captures attitude evolution through rotational kinematics. The bias states account for the slowly varying sensor offsets that develop over time due to temperature changes, component aging, and other environmental factors. Together, these quantities form a complete minimal representation of the navigation state used in inertial systems, forming the basis for the continuous time kinematic model that governs their time evolution.  
\\ \\
The inertial navigation model therefore describes the kinematic evolution of the vehicles position, velocity, and attitude as driven by inertial sensor measurements. The foundation of this formulation is the deterministic, noise free relationship between translational and rotational motion, which forms the core of all INS propagation models.  
\\ \\
In the ideal case without any sensor errors, the motion of the sensor can be expressed as
$$
\begin{aligned}
    \dot{\mathbf{p}}_{s/O}^{n} &= \mathbf{v}_{s/O}^{n} \\
    \dot{\mathbf{v}}_{s/O}^{n} &= R_b^n(\mathbf{q})\,(R_s^b\,\mathbf{a}_m) + \mathbf{g}^n \\
    \dot{\mathbf{q}} &= \tfrac{1}{2}\,\mathbf{q} \otimes (R_s^b\,\boldsymbol{\omega}_m)
\end{aligned}
$$
where $\mathbf{p}_{s/O}^{n}$ and $\mathbf{v}_{s/O}^{n}$ are the position and velocity of the sensor with respect to an inertial reference frame like NED, $R_b^n(\mathbf{q})$ is the rotation matrix transforming vectors from the body frame to the navigation frame, and $R_s^b$ is the rotation matrix from the sensor to body frame. The measured quantities $\mathbf{a}_m$ and $\boldsymbol{\omega}_m$ are the specific force and angular rate readings provided by the IMU, expressed in the sensors own coordinate frame. These measurements are therefore related to the true body frame motion through the sensors fixed mounting orientation, such that $\mathbf{a}_m = \mathbf{a}_{s/b}^{s}$ and $\boldsymbol{\omega}_m = \boldsymbol{\omega}_{s/b}^{s}$.  
\\ \\
In practice, the inertial sensors are rarely placed exactly at the vehicles center of gravity (COG). When the sensor is offset by a position vector $\mathbf{r}_{s/b}^{b}$ from the body frame origin usually defined at the COG, the accelerations measured by the IMU no longer correspond directly to the translational accelerations of the vehicle. Instead, the rotational motion of the body introduces additional terms due to centripetal and tangential accelerations. The total acceleration of the sensor can then be expressed as
$$
\mathbf{a}_{s/O}^{b} = \mathbf{a}_{b/O}^{b} + \dot{\boldsymbol{\omega}}_{b/O}^{b} \times \mathbf{r}_{s/b}^{b} + \boldsymbol{\omega}_{b/O}^{b} \times (\boldsymbol{\omega}_{b/O}^{b} \times \mathbf{r}_{s/b}^{b})
$$
where $\mathbf{a}_{b/O}^{b}$ is the linear acceleration of the body origin, $\boldsymbol{\omega}_{b/O}^{b}$ is the angular velocity of the body, and $\dot{\boldsymbol{\omega}}_{b/O}^{b}$ its angular acceleration. The second term represents the tangential acceleration component caused by rotational acceleration, while the third term corresponds to the centripetal acceleration arising from constant rotation. These effects are collectively known as the \textit{``lever arm effect''}, and they cause the IMU to measure accelerations that differ from those experienced at the COG.  
\\ \\
Accurately compensating for the lever arm effect requires knowledge of the exact mounting offset $\mathbf{r}_{s/b}^{b}$ and precise estimates of both angular velocity and angular acceleration. The latter is particularly difficult to obtain in real time, as most MEMS-based IMU (like microAmpere IMU) do not measure angular acceleration directly. Numerical differentiation of gyroscope signals amplifies measurement noise and introduces additional uncertainty, making real-time compensation computationally challenging and often unreliable.  
\\ \\
Therefore, in this master thesis work, it is assumed that the inertial sensors are mounted close to the vehicles center of gravity. For the microAmpere ASV, this assumption is valid since the IMU is physically located near the COG within the central electronics housing. As a result, the lever arm terms become negligible, and the simplified kinematic model given above provides an accurate and practical representation of the vehicle motion. This assumption greatly simplifies the mathematical model while retaining sufficient fidelity for navigation and SLAM applications.  
\\ \\
Under this assumption, the IMU can be considered approximately coincident with the body frame origin, effectively treating the sensor as rigidly aligned with the vehicles center of gravity. The motion model can then be expressed directly in terms of the body frame as
$$
\begin{aligned}
    \dot{\mathbf{p}}_{b/O}^{n} &= \mathbf{v}_{b/O}^{n} \\
    \dot{\mathbf{v}}_{b/O}^{n} &= R_b^n(\mathbf{q})\,\mathbf{a}_m + \mathbf{g}^n \\
    \dot{\mathbf{q}} &= \tfrac{1}{2}\,\mathbf{q} \otimes \boldsymbol{\omega}_m
\end{aligned}
$$
which defines the nominal, noise free kinematic model used in this work. It captures the translational and rotational motion of the ASV based solely on the IMUs accelerometer and gyroscope measurements, with attitude expressed through quaternion integration for smooth and singularity free orientation propagation.
\\ \\
In practice, the accelerometer and gyroscope measurements are not perfect and are affected by sensor biases and measurement noise. These imperfections cause the estimated position and attitude to drift over time, as small integration errors accumulate during motion. The measured sensor outputs can therefore be modeled as
$$
\begin{aligned}
    \mathbf{a}_m &= R^{T}(q_t)\,(\mathbf{a_t} - \mathbf{g}) + \mathbf{a}_{bt} + \mathbf{a}_{n} \\
    \boldsymbol{\omega}_m &= \boldsymbol{\omega}_t + \mathbf{\omega}_{bt} + \mathbf{\omega}_{n}
\end{aligned}
$$
where $\mathbf{a}_m$ and $\boldsymbol{\omega}_m$ are the measured accelerations and angular rates, $\mathbf{a}_{bt}$ and $\mathbf{\omega}_{bt}$ denote slowly varying sensor biases, and $\mathbf{a}_{n}$ and $\mathbf{\omega}_{n}$ are zero mean Gaussian noise processes representing measurement uncertainty. The reason for choosing zero mean Gaussian noise processes representation is just to make calculations easier to manage, and usually inertial sensor noise can be approximated to a pure gaussian.  
\\ \\
While the previous kinematic model assumed ideal, noise free measurements, real inertial sensors inherently produce signals corrupted by both bias and random noise. These effects introduce long term drift in velocity, position, and orientation estimates if uncorrected. To account for this, the noise and bias terms are explicitly included in the dynamic equations.  
\\ \\
By substituting the noisy sensor models into the noise free kinematics, we obtain the full continuous time INS motion model that represents the true system behavior under realistic sensor conditions:
$$
\begin{aligned}
    \dot{\mathbf{p}}_{b/O}^{n} &= \mathbf{v}_{b/O}^{n} \\
    \dot{\mathbf{v}}_{b/O}^{n} &= R_b^n(\mathbf{q})\,(\mathbf{a}_m - \mathbf{a}_{bt} - \mathbf{a}_n) + \mathbf{g}^n \\
    \dot{\mathbf{q}} &= \tfrac{1}{2}\,\mathbf{q} \otimes (\boldsymbol{\omega}_m - \mathbf{\omega}_{bt} - \mathbf{\omega}_n)
\end{aligned}
$$
where $R_b^n(\mathbf{q})$ transforms body frame accelerations into the navigation frame according to the current orientation quaternion.  
\\ \\
This formulation captures the complete inertial motion propagation process, accounting for both deterministic motion and stochastic sensor effects. It forms the basis for all subsequent modeling of bias dynamics and sensor noise processes.
\\ \\
Inertial measurement systems such as accelerometers and gyroscopes inherently suffer from drift over time, as they measure relative motion rather than absolute quantities. Small integration errors in acceleration and angular rate accumulate, causing increasing uncertainty in position and attitude estimates. To account for this phenomenon, bias terms are introduced to represent slow, time varying offsets in the sensor readings. Accurate bias modeling is essential for achieving reliable and stable navigation performance, particularly for low cost MEMS-based sensors where bias instability is a dominant error source.  
\\ \\
Sensor biases are typically represented as stochastic processes that evolve gradually over time. Several modeling approaches exist, each offering a trade off between realism and complexity. The simplest approach is the constant bias model,
$$
    \dot{\mathbf{b}} = 0
$$
which assumes a fixed, time invariant offset in the sensor readings. Although straightforward, this model is generally insufficient for practical applications since real world sensors exhibit continuous drift due to thermal, mechanical, and electronic variations.  
\\ \\
A more flexible formulation is the Wiener process, or better known as random walk model,
$$
    \dot{\mathbf{b}} = \mathbf{b}_n
$$
where $\mathbf{b}_n$ is a zero mean white noise process. This allows the bias to evolve stochastically over time, capturing unmodeled variations in sensor output. However, this model tends to be overly conservative, as the bias uncertainty grows without bound, making it unsuitable for long duration navigation where bounded behavior is required.  
\\ \\
A more realistic representation for modern inertial sensors is the first-order Gauss-Markov process,
$$
    \dot{\mathbf{b}} = -\tfrac{1}{\tau}\mathbf{b} + \mathbf{b}_n
$$
where $\tau$ is the correlation time constant that defines how quickly the bias decays toward zero. This model captures both short term fluctuations and long term drift, providing a balanced compromise between physical accuracy and numerical stability. It ensures that bias uncertainty remains bounded while reflecting the natural stochastic variation of real IMU behavior.  
\\ \\
Accordingly, the accelerometer and gyroscope biases are modeled as first order Gauss-Markov processes:
$$
\begin{aligned}
    \dot{\mathbf{a}}_{bt} &= -p_{\mathbf{a}b}\,I\,\mathbf{a}_{bt} + \mathbf{a}_{w} \\
    \dot{\mathbf{\omega}}_{bt} &= -p_{\mathbf{\omega}b}\,I\,\mathbf{\omega}_{bt} + \mathbf{\omega}_{w}
\end{aligned}
$$
where $p_{\mathbf{a}b} = 1/\tau_a$ and $p_{\mathbf{\omega}b} = 1/\tau_g$ denote the inverse correlation times for the accelerometer and gyroscope bias models, respectively. The terms $\mathbf{a}_{w}$ and $\mathbf{\omega}_{w}$ represent zero mean white noise processes driving the bias evolution.  
\\ \\
This bias modeling framework provides a physically consistent and numerically stable representation of inertial sensor drift. By incorporating these bias dynamics, the INS model can accurately describe both the deterministic vehicle motion and the stochastic effects introduced by the inertial sensors.  
\\ \\
With the inclusion of bias dynamics, the complete continuous time true state kinematic model of the system is expressed as
$$
\begin{aligned}
    \dot{\mathbf{p}}_{b/O}^{n} &= \mathbf{v}_{b/O}^{n} \\
    \dot{\mathbf{v}}_{b/O}^{n} &= R_b^n(\mathbf{q})\,(\mathbf{a}_m - \mathbf{a}_{bt} - \mathbf{a}_n) + \mathbf{g}^n \\
    \dot{\mathbf{q}} &= \tfrac{1}{2}\,\mathbf{q} \otimes (\boldsymbol{\omega}_m - \mathbf{\omega}_{bt} - \mathbf{\omega}_n) \\
    \dot{\mathbf{a}}_{bt} &= -p_{\mathbf{a}b}\,I\,\mathbf{a}_{bt} + \mathbf{a}_{w} \\
    \dot{\mathbf{\omega}}_{bt} &= -p_{\mathbf{\omega}b}\,I\,\mathbf{\omega}_{bt} + \mathbf{\omega}_{w}
\end{aligned}
$$
This formulation constitutes the complete nonlinear inertial motion model, capturing both the deterministic kinematics of the vehicle and the stochastic processes that characterize real sensor behavior.
\\ \\
The complete INS model previously presented captures the full physical and stochastic behavior of the vehicle, incorporating both deterministic motion and random processes such as measurement noise and bias drift. While this representation accurately reflects real sensor behavior, it is often desirable to isolate the deterministic part of the dynamics to describe the nominal system behavior. This simplified formulation, known as the nominal state kinematic model, represents how the system would evolve in the absence of stochastic disturbances. It provides a clear and noise-free description of the vehicles motion based purely on inertial measurements, forming the foundation for state propagation in navigation and estimation frameworks.  
\\ \\
The nominal model assumes that all noise components are zero, leaving only the essential deterministic relationships between position, velocity, attitude, and sensor biases. In this formulation, the accelerometer and gyroscope measurements are treated as control inputs driving the systems motion. This perspective simplifies the dynamics while preserving the physical structure of the original model, allowing efficient and accurate propagation of the navigation states between external aiding updates.  
\\ \\
By removing the stochastic noise terms from the true state equations, the nominal state kinematic model becomes
$$
    \dot{\mathbf{x}} = f(\mathbf{x}, \mathbf{u}) =
    \begin{bmatrix}
        \dot{\mathbf{p}}_{b/O}^{n} \\
        \dot{\mathbf{v}}_{b/O}^{n} \\
        \dot{\mathbf{q}} \\
        \dot{\mathbf{a}}_b \\
        \dot{\mathbf{\omega}}_b
    \end{bmatrix}
    =
    \begin{bmatrix}
        \mathbf{v}_{b/O}^{n} \\
        R_b^n(\mathbf{q})\,(\mathbf{a}_m - \mathbf{a}_{b}) + \mathbf{g}^n \\
        \tfrac{1}{2}\,\mathbf{q} \otimes (\boldsymbol{\omega}_m - \mathbf{\omega}_{b}) \\
        -p_{\mathbf{a}b}\,I\,\mathbf{a}_b \\
        -p_{\mathbf{\omega}b}\,I\,\mathbf{\omega}_b
    \end{bmatrix}
$$
where
$$
    \mathbf{x} =
    \begin{bmatrix}
        \mathbf{p}_{b/O}^{n} & \mathbf{v}_{b/O}^{n} & \mathbf{q} & \mathbf{b}_a & \mathbf{b}_g
    \end{bmatrix}^\top,
    \qquad
    \mathbf{u} =
    \begin{bmatrix}
        \mathbf{a}_m & \boldsymbol{\omega}_m
    \end{bmatrix}^\top
$$
Here, $\mathbf{p}^n$ and $\mathbf{v}^n$ represent the vehicles position and velocity expressed in the navigation frame, $\mathbf{q}^{bn}$ denotes the attitude quaternion defining the rotation from the body to the navigation frame, and $\mathbf{b}_a$ and $\mathbf{b}_g$ represent the accelerometer and gyroscope biases, respectively. The parameters $p_a = 1/\tau_a$ and $p_g = 1/\tau_g$ correspond to the inverse correlation times governing the exponential decay rates of the bias dynamics.  
\\ \\
This nominal model represents the idealized, noise free motion of the vehicle as inferred from IMU measurements. It captures the deterministic propagation of position, velocity, attitude, and bias under perfect sensor conditions, serving as the baseline dynamic model for integrated navigation systems.  
\\ \\
By separating deterministic kinematics from stochastic sensor effects, the INS formulation provides a structured foundation for real-time state estimation. This nominal model is particularly suited for use within an Error State Kalman Filter (ESKF), where it defines the expected system evolution between measurement updates and enables correction using external aiding sources such as GNSS or magnetometers.  
\\ \\
Overall, the INS based motion model offers a compact and computationally efficient representation of the ASVs dynamics. While it omits detailed hydrodynamic effects, its balance of simplicity, generality, and compatibility with sensor fusion frameworks makes it a robust and practical choice for autonomous surface vehicle navigation and SLAM applications.



\subsubsection{GNSS as Aiding Measurement Model}
This subsection defines the aiding model using dual-antenna GNSS measurements. It should:
\begin{itemize}
    \item Describe how each antenna provides position measurements in the NED frame.
    \item Derive the measurement model for position:
    $$
    \mathbf{z}_p = \mathbf{p} + \mathbf{n}_p
    $$
    \item Explain heading estimation from the baseline vector between antennas.
    \item Show how measurements are fused in the ESKF as linear updates.
    \item Discuss measurement noise modeling, covariance tuning, and outlier rejection.
\end{itemize}
This section establishes the absolute reference aiding used to correct INS drift and provide global consistency.



\subsubsection{Model Comparison and Selection}
This subsection compares the Fossen marine craft model and the INS-based kinematic model. It should:
\begin{itemize}
    \item Discuss the trade-offs between dynamic completeness and model simplicity.
    \item Argue that while Fossen’s model provides high physical accuracy, it requires extensive parameter identification (added mass, damping, hydrodynamic coefficients).
    \item Highlight that the INS model, though simplified, is sufficient for SLAM and data processing tasks.
    \item Emphasize that for building a local map and performing state estimation with aiding sensors, the INS model provides an optimal balance of accuracy and practicality.
    \item Conclude that a full system identification for marginal performance gain is not justified for this application.
\end{itemize}
This section justifies the modeling approach used in this thesis, positioning the INS-based model as the preferred choice for SLAM-oriented estimation.

