\subsection{Rigid Body Kinematics}
This section presents the mathematical relationships describing the translational and rotational motion of a rigid body. These quantities form the foundation of inertial navigation, control, and sensor fusion systems, where the position, orientation, velocity, and acceleration must be consistently defined and transformed between coordinate frames.  



\subsubsection{Position and Orientation}
The position of a rigid body is represented by the vector $\mathbf{p}_b^n$, which denotes the location of the body frame origin expressed in the navigation (NED) frame. The orientation of the body is represented by the rotation matrix $R_b^n \in \mathrm{SO}(3)$, which maps a vector from the body frame $\{b\}$ to the navigation frame $\{n\}$.  
\\ \\
The combined rigid body pose is described by the homogeneous transformation matrix $T_b^n \in \mathrm{SE}(3)$:
$$
    T_b^n =
    \begin{bmatrix}
        R_b^n & \mathbf{p}_b^n \\
        0 & 1
    \end{bmatrix}
$$
The kinematic relationship between the time derivative of position and the linear velocity is then given by
$$
    \dot{\mathbf{p}}_b^n = \mathbf{v}_b^n
$$
where $\mathbf{v}_b^n$ is the velocity of the body origin expressed in the navigation frame.  



\subsubsection{Angular Velocity and Euler Angle Relationship}
The angular velocity vector $\boldsymbol{\omega}^b = [p, q, r]^\top$ represents the instantaneous rotation rate of the body about its own axes, roll rate $p$ about the $x_b$-axis (North), pitch rate $q$ about the $y_b$-axis (East), and yaw rate $r$ about the $z_b$-axis (Down).  
\\ \\
When the body orientation is represented by ZYX Euler angles (yaw $\psi$, pitch $\theta$, roll $\phi$), the time derivative of the Euler angles relates to the body angular velocity through
$$
    \begin{bmatrix}
        \dot{\phi} \\ \dot{\theta} \\ \dot{\psi}
    \end{bmatrix}
    = T(\phi, \theta)\boldsymbol{\omega}^b
$$
where the transformation matrix $T(\phi, \theta)$ for the NED convention is defined as
$$
    T(\phi, \theta) =
    \begin{bmatrix}
        1 & \sin\phi\tan\theta & \cos\phi\tan\theta \\
        0 & \cos\phi & -\sin\phi \\
        0 & \sin\phi/\cos\theta & \cos\phi/\cos\theta
    \end{bmatrix}
$$
This maps the body angular velocity to the Euler angle rates according to the NED convention.  
\\ \\
The inverse relationship, expressing body angular velocity from Euler angle derivatives, is given by
$$
    \boldsymbol{\omega}^b =
    \begin{bmatrix}
        p \\ q \\ r
    \end{bmatrix}
    = T^{-1}(\phi, \theta)
    \begin{bmatrix}
        \dot{\phi} \\ \dot{\theta} \\ \dot{\psi}
    \end{bmatrix}
$$
where
$$
    T^{-1}(\phi, \theta) =
    \begin{bmatrix}
        1 & 0 & -\sin\theta \\
        0 & \cos\phi & \sin\phi\cos\theta \\
        0 & -\sin\phi & \cos\phi\cos\theta
    \end{bmatrix}
$$
This formulation aligns with the NED coordinate convention and ensures correct mapping between angular velocities and Euler angle derivatives. The transformation matrix becomes singular at $\theta = \pm 90^\circ$, corresponding to gimbal lock, where $\tan\theta$ diverges.  



\subsubsection{Angular Velocity and Quaternion Relationship}
Quaternions provide a compact and singularity free representation of rotation. A unit quaternion $q = [q_0, q_1, q_2, q_3]^\top$ consists of one scalar and three vector components, often written as
$$
    q = 
    \begin{bmatrix}
        q_0 \\ \mathbf{q}_v
    \end{bmatrix}
    =
    \begin{bmatrix}
        q_0 \\ q_1 \\ q_2 \\ q_3
    \end{bmatrix}
$$
where $q_0$ is the scalar part and $\mathbf{q}_v = [q_1, q_2, q_3]^\top$ is the vector part.  
The quaternion represents a rotation of angle $\theta$ around the unit axis $\hat{\mathbf{u}}$:
$$
    q =
    \begin{bmatrix}
        \cos\frac{\theta}{2} \\ \hat{\mathbf{u}}\sin\frac{\theta}{2}
    \end{bmatrix}
$$
The time evolution of the quaternion is governed by the angular velocity vector $\boldsymbol{\omega}^b = [p, q, r]^\top$ measured in the body frame. The quaternion kinematics are given by
$$
    \dot{q} = \frac{1}{2}\Omega(\boldsymbol{\omega}^b) q
$$
where $\Omega(\boldsymbol{\omega}^b)$ is the quaternion rate matrix:
$$
    \Omega(\boldsymbol{\omega}^b) =
    \begin{bmatrix}
        0 & -p & -q & -r \\
        p & 0 & r & -q \\
        q & -r & 0 & p \\
        r & q & -p & 0
    \end{bmatrix}
$$
This equation directly integrates body angular velocity into quaternion rate, forming the basis for attitude propagation in inertial navigation systems. The quaternion must remain normalized, i.e. $\|q\| = 1$, to represent a valid rotation. To enforce this during numerical integration, the quaternion is periodically renormalized as
$$
    q = \frac{q}{\|q\|}
$$
The corresponding rotation matrix from body to navigation frame is obtained as
$$
    R_b^n(q) =
    \begin{bmatrix}
        1 - 2(q_2^2 + q_3^2) & 2(q_1 q_2 - q_0 q_3) & 2(q_1 q_3 + q_0 q_2) \\
        2(q_1 q_2 + q_0 q_3) & 1 - 2(q_1^2 + q_3^2) & 2(q_2 q_3 - q_0 q_1) \\
        2(q_1 q_3 - q_0 q_2) & 2(q_2 q_3 + q_0 q_1) & 1 - 2(q_1^2 + q_2^2)
    \end{bmatrix}
$$
This rotation matrix provides the mapping between the body and navigation frames, analogous to $R_b^n$ obtained from Euler angles, but without the gimbal lock problem.  
\\ \\
In practice, quaternion based orientation integration using $\dot{q} = \frac{1}{2}\Omega(\boldsymbol{\omega}^b)q$ is preferred for real-time attitude propagation, while the equivalent rotation matrix or Euler angles are extracted only when needed for visualization or control.



\subsubsection{Linear Velocity Relationship}
The linear velocity of any point $P$ on a rigid body can be related to the velocity of a reference point $O$ on the same body using
$$
    \mathbf{v}_P = \mathbf{v}_O + \boldsymbol{\omega} \times \mathbf{r}_{OP}
$$
where $\mathbf{v}_O$ is the translational velocity of the reference point (often the center of mass or body origin), $\boldsymbol{\omega} = [p, q, r]^\top$ is the angular velocity vector of the body expressed in the body frame, and $\mathbf{r}_{OP}$ is the position vector from $O$ to $P$ expressed in the same frame. The cross product $\boldsymbol{\omega} \times \mathbf{r}_{OP}$ represents the additional linear velocity at point $P$ due to the body's rotation.  
\\ \\
This relation shows that even if the body origin $O$ has a purely translational motion, other points on the body experience additional velocity components depending on the rotational motion and their position relative to $O$. For instance, sensors or antennas mounted away from the vehicles center of gravity will experience slightly different linear velocities when the vehicle rotates.
\\ \\
In navigation systems, it is often necessary to express the velocity in a common reference frame. The linear velocity of the body, measured in the body frame, can be transformed to the navigation frame (typically NED) through the rotation matrix $R_b^n$ as
$$
    \mathbf{v}_b^n = R_b^n \mathbf{v}_b^b
$$
where $\mathbf{v}_b^b$ is the velocity vector expressed in the body frame, and $\mathbf{v}_b^n$ is the same velocity expressed in the navigation frame. The inverse transformation is given by
$$
    \mathbf{v}_b^b = (R_b^n)^\top \mathbf{v}_b^n
$$
since $(R_b^n)^\top = R_n^b$.  
\\ \\
This transformation is essential for integrating IMU and GNSS data. IMU provide specific force and angular velocity in the body frame, while GNSS delivers velocity and position in the global or navigation frame. Expressing both in the same frame allows consistent state estimation and sensor fusion. In practice, $R_b^n$ is obtained either from Euler angles or quaternions, ensuring accurate frame alignment between inertial and navigation states.



\subsubsection{Angular Acceleration in Euler Representation}
The angular acceleration $\boldsymbol{\alpha}^b$ describes the rate of change of the body's angular velocity vector $\boldsymbol{\omega}^b = [p, q, r]^\top$, which represents the instantaneous rotation of the body about its own $x$, $y$, and $z$ axes. Physically, $\boldsymbol{\alpha}^b$ captures how fast the rotational motion itself is changing, and it directly relates to the torques acting on the body through the rotational dynamics equations.  
\\ \\
When the body orientation is represented using Euler angles $(\phi, \theta, \psi)$, the angular acceleration can be obtained by differentiating the Euler angle to angular velocity relationship:
$$
    \boldsymbol{\alpha}^b = T^{-1}(\phi, \theta, \psi)\ddot{\boldsymbol{\Theta}} + \dot{T}^{-1}(\phi, \theta, \psi)\dot{\boldsymbol{\Theta}}
$$
where $\dot{\boldsymbol{\Theta}} = [\dot{\phi}, \dot{\theta}, \dot{\psi}]^\top$ and $\ddot{\boldsymbol{\Theta}} = [\ddot{\phi}, \ddot{\theta}, \ddot{\psi}]^\top$ are the first and second derivatives of the Euler angles. The first term represents the direct contribution from angular rate changes, while the second term captures coupling effects caused by the nonlinearity of rotational kinematics, for instance, when pitch or roll rates influence yaw acceleration.  
\\ \\
In differential form, the time derivative of angular velocity is expressed as
$$
    \dot{\boldsymbol{\omega}}^b = \boldsymbol{\alpha}^b
$$
which defines angular acceleration in the same frame as $\boldsymbol{\omega}^b$. To express angular acceleration in another reference frame, such as the navigation frame, it can be transformed using the rotation matrix:
$$
    \boldsymbol{\alpha}^n = R_b^n \boldsymbol{\alpha}^b
$$
where $R_b^n$ is the rotation from the body frame to the navigation (NED) frame.  
\\ \\
In practical navigation and control systems, the angular acceleration $\boldsymbol{\alpha}^b$ is rarely measured directly. It is usually approximated by differentiating the body angular velocity $\boldsymbol{\omega}^b$ obtained from gyroscopes or inferred from rigid body dynamics. Maintaining consistent frame alignment between body and navigation frames is essential for accurate attitude propagation and torque computation.



\subsubsection{Angular Acceleration in Quaternion Representation}
The quaternion representation provides a singularity free alternative to Euler angles for describing rotational motion. The unit quaternion $q = [q_0, q_1, q_2, q_3]^\top$ defines the orientation of the body frame relative to the navigation frame, where $q_0$ is the scalar part and $[q_1, q_2, q_3]^\top$ is the vector part.  
\\ \\
The time evolution of the quaternion is governed by the rotational kinematics equation:
$$
    \dot{q} = \frac{1}{2}\Omega(\boldsymbol{\omega}^b)q
$$
where $\boldsymbol{\omega}^b = [p, q, r]^\top$ is the angular velocity in the body frame and $\Omega(\boldsymbol{\omega}^b)$ is the quaternion rate matrix:
$$
    \Omega(\boldsymbol{\omega}^b) =
    \begin{bmatrix}
        0 & -p & -q & -r \\
        p &  0 &  r & -q \\
        q & -r &  0 &  p \\
        r &  q & -p &  0
    \end{bmatrix}
$$
Differentiating this expression gives the quaternion based angular acceleration:
$$
    \ddot{q} = \frac{1}{2}\Omega(\boldsymbol{\alpha}^b)q + \frac{1}{2}\Omega(\boldsymbol{\omega}^b)\dot{q}
$$
which captures both the direct contribution of angular acceleration $\boldsymbol{\alpha}^b$ and the coupling effect from the current angular velocity.  
\\ \\
In vector form, the time derivative of angular velocity is expressed as
$$
    \dot{\boldsymbol{\omega}}^b = \boldsymbol{\alpha}^b
$$
which defines angular acceleration in the same frame as $\boldsymbol{\omega}^b$. To express angular acceleration in another reference frame, such as the navigation frame, it can be transformed using the rotation matrix:
$$
    \boldsymbol{\alpha}^n = R_b^n \boldsymbol{\alpha}^b
$$
where $R_b^n$ is the rotation matrix from body to navigation frame.
\\ \\
To ensure $q$ remains a valid rotation, it must satisfy the unit norm constraint:
$$
    \|q\| = 1
$$
which is typically enforced by periodic normalization during numerical integration like so:
$$
    q = \frac{q}{\|q\|}
$$  
This quaternion based formulation is computationally efficient and avoids the gimbal lock issue inherent in Euler angle representations. For this reason, quaternion propagation is widely preferred for representing orientation and modeling angular dynamics in navigation and control systems.



\subsubsection{Linear Acceleration}
The linear acceleration of a point $P$ on a rigid body consists of translational, tangential, centripetal, and coriolis components. It is expressed as
$$
    \mathbf{a}_P =
    \mathbf{a}_O
    + \boldsymbol{\alpha} \times \mathbf{r}_{OP}
    + \boldsymbol{\omega} \times (\boldsymbol{\omega} \times \mathbf{r}_{OP})
    + 2\boldsymbol{\omega} \times \mathbf{v}_{rel},
$$
where $\mathbf{a}_O$ is the acceleration of the reference point (often the body origin or center of mass), $\boldsymbol{\alpha}$ is the angular acceleration vector, $\boldsymbol{\omega}$ is the angular velocity vector, $\mathbf{r}_{OP}$ is the position vector from the reference point $O$ to the point $P$, and $\mathbf{v}_{rel}$ is the velocity of $P$ relative to the rotating frame. The first term represents the translational acceleration of the reference point, the second term the tangential acceleration caused by the change in rotational rate, the third term the centripetal acceleration directed toward the axis of rotation, and the fourth term the coriolis acceleration arising from relative motion within the rotating frame.  
\\ \\
When the body fixed frame is used, the total acceleration of a point expressed in the navigation (NED) frame is related through the rotation matrix as
$$
    \mathbf{a}_b^n = R_b^n \mathbf{a}_b^b,
$$
and conversely,
$$
    \mathbf{a}_b^b = (R_b^n)^\top \mathbf{a}_b^n.
$$
These transformations ensure consistent representation between body frame sensor measurements and navigation frame quantities.  
\\ \\
In inertial navigation systems, accelerometers measure the specific force, which is the total acceleration excluding gravity, given by
$$
    \mathbf{f}^b = \mathbf{a}_b^b - R_n^b \mathbf{g}^n,
$$
where $\mathbf{g}^n = [0, 0, g]^\top$ is the gravity vector in the NED frame. The specific force $\mathbf{f}^b$ is the quantity directly measured by IMUs and forms the basis for estimating velocity and position through numerical integration.

