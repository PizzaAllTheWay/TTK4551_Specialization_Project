\subsection{Goal}
\todo{TAP: The thesis is already starting to get long. At Marine Department, they have a limit of approx 80 pages for a master thesis (excluding appendices and references). We are now talking about a pre-project. Discuss this with Damiano, I'm not sure about Cybernetics, but anyway, when comming to the master thesis, being precise and concise will be key.}
\todo{TAP: Goal and motivation are natural in the introduction chapter, but not sure if Side Scan chapter is? In addition, this chapter should include some background (possible together with motivation), objectives (possible together with goal), contributions, structure of the thesis.  Have a look at other master thesis for inspiration.}
\todo[inline]{Maybe I should resturcture the Introduction Goal and Motivation maybe?...}
\todo[inline]{Also add the FN bærekrafts mål I suppose?...}
\noindent
This specialization project focuses on navigation and Simultaneous Localization and Mapping (SLAM) for marine robots, with an emphasis on Side Scan Sonar based SLAM. The main objective is to study, reimplement, and optimize the core components of modern SSS SLAM pipelines to achieve real-time performance on real world AUV datasets, such as those collected from NTNU's AUR Lab platforms (for example LAUV Harald). 
\\ \\
The work builds upon the 2023 masters thesis by Haraldstad \cite{side_scan_sonar_master_thesis} and recent research on side scan sonar landmark detection \cite{side_scan_sonar_paper}, which are heavily inspired by the earlier work of Hogstad, Bjørnar Reitan \cite{side_scan_sonar_master_thesis_old}. The projects scope is to implement and benchmark the essential components required for real-time SSS SLAM operation, validate them on existing datasets, and demonstrate that the processing pipeline can operate at or above the sonars frame rate while maintaining high mapping accuracy.
\\ \\
A second phase of the project focuses on embedded deployment on a autonomous surface vessel (ASV). The goal is to port and integrate the validated core pipeline on ship borne hardware, assess side scan sonar performance in coastal waters, and shift the work from algorithm design to practical system integration. This phase will use NTNU's MicroAmpere ASV.


