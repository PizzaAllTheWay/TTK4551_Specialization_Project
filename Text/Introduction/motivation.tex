\subsection{Motivation}
Marine robots must make decisions without constant human input, often far from easy access. They need to know where they are and what surrounds them to move safely and do useful work. A prior map helps, but the ocean changes over time. Currents, waves, moving vessels, new structures, and shifting seabeds make static maps go out of date. \gls{GNSS} data is weak or unavailable underwater, and dead reckoning drifts over time. Even in coastal areas, fjords and valleys can block signals, and ships operating in shallow water want to avoid the bottom with good margins. Because of this, robots need to build and update their own map while they localize in it, this is a \gls{SLAM} problem. \gls{SLAM} fuses sensors, handles loop closures to correct drift, and provides a consistent pose and an up to date map during the mission. Sonar is a key sensor for robust navigation underwater, and side scan sonar in particular provides wide swath, high resolution imagery that can reveal seafloor structure and landmarks suitable for data association. Focusing on \gls{SSS SLAM} lets us study a concrete, relevant problem, how to extract stable features from side scan sonar, associate them across passes, close loops, and feed this information into a \gls{SLAM} back-end that remains fast and stable over long runs. This matters both for precise mapping and for safe, efficient navigation when external positioning is unreliable.
\\ \\
The motivation is to run state of the art \gls{SLAM} on real hardware in real time and see how it actually performs. We want to measure accuracy, robustness, and runtime on real \gls{AUV} data, not just on papers or simulations, and learn what changes are needed to make it reliable at sea. In short, take modern \gls{SLAM}, make it work on the robot, and judge it by field results.