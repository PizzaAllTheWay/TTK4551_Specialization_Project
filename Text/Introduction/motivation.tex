\subsection{Motivation}
Reliable maritime navigation requires onboard estimation and mapping when external positioning is weak or unavailable. This project focuses on side scan sonar based simultaneous localization and mapping, SSS SLAM. SSS SLAM uses side scan sonar to estimate the vehicle pose while building a seafloor map at the same time. In SSS SLAM, sonar images drive feature extraction and data association, loop closures correct drift, and the SLAM back end fuses all measurements into a consistent trajectory and map. Marine robots operate with limited access and often far from support. They must know where they are and what surrounds them to move safely and do useful work. Static charts help, but the ocean changes over time. Currents, waves, moving vessels, new structures, and shifting seabeds make static maps go out of date. GNSS is weak or unavailable underwater, and dead reckoning drifts. Even in coastal areas, terrain can block signals, and shallow water operations require safe margins to the bottom. These factors make SSS SLAM a practical path to robust navigation and mapping.
\\ \\
Prior work \cite{side_scan_sonar_master_thesis} presented a pipeline for SSS SLAM but did not reach real time performance. Field deployment needs real time operation on real data with measured accuracy and robustness. The first motivation is to deliver a lean real time SSS SLAM implementation and to quantify performance on the same dataset used previously, so the results are directly comparable.
\\ \\
The second motivation is to move from offline studies to reliable system behavior at sea. The plan is to measure accuracy, robustness, and runtime on recorded AUV data, then prepare the pipeline for embedded use on a autonomous surface vessel (ASV). This shifts effort from algorithm design to practical integration on real hardware, including time synchronization, calibration, and stable runtime.


