\subsection{Background}
Coastal navigation depends on accurate environmental perception and robust mapping, especially in shallow or dynamic areas where shifting seabed conditions and variable coastlines make grounding avoidance a critical task. While GNSS normally provides global positioning, its performance degrades near coastal infrastructure due to shadowing and multipath. Autonomous surface vessels therefore require onboard estimation and mapping that remain reliable even when external positioning is weak or unavailable. Side scan sonar is well suited for this role, as acoustic sensing penetrates turbid water and provides high resolution seafloor information independent of visibility. In recent years, Side Scan Sonar based SLAM (SSS SLAM) has become an attractive approach for estimating the vessel trajectory while reconstructing local seabed structure, supporting both safe navigation and situational awareness in coastal environments.
\\ \\
Prior research, including paper on side scan sonar landmark detection \cite{side_scan_sonar_paper} and Haraldstads 2023 master thesis on Side Scan Sonar SLAM (SSS SLAM) \cite{side_scan_sonar_master_thesis}, which are heavily inspired by the earlier work of Hogstad \cite{side_scan_sonar_master_thesis_old}, demonstrated the feasibility of using side scan sonar for feature extraction, data association, and landmark based mapping. However, most existing pipelines operate offline, struggle with real-time performance. Modern factor graph SLAM systems, such as those developed by Leonard, Kaess, and the GTSAM community, provide a strong theoretical foundation for building efficient, incremental SLAM architectures suitable for embedded robotic platforms. This project builds upon those foundations and focuses on adapting, simplifying, and optimizing SSS SLAM for deployment on NTNU's MicroAmpere ASV (Autonomous Surface Vessel).
